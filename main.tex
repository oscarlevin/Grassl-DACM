



\section{FOUR (OR EIGHT) DISTRIBUTION PROBLEMS}

The formulas derived earlier for the number of \emph{onto} functions f
from a domain \(\{ x_{1},\ x_{2},\ldots,\ x_{m}\}\) to a codomain
\(\{ y_{1},\ y_{2},\ldots,\ y_{n}\}\) can be used as a model for the
following distribution problem: in how many ways can you place n
different objects into k different bins, not allowing empty bins? Some
texts use the word distinguishable instead of different. We will also
use the word similar to connote the same meaning as indistinguishable.

The concept of \emph{compositions} of n into k parts likewise serves as
a model for the following distribution problem: in how many ways can you
place n similar objects into k different boxes either allowing empty
boxes or not? As this second class of problem is a little easier let's
analyze it first.

\begin{theorem} The number of ways of placing n similar objects into k different
bins not allowing empty bins is \(\binom{n - 1}{k - 1}
\). Denote the k different bins by
\(x_{1},\ x_{2},\ldots,\ x_{k}\). The number of solutions to
\(x_{1} + x_{2} + \ \ldots + x_{k} = n\) where each \(x_{i} \geq 1\) is
the number of such distributions.
\end{theorem}

\begin{theorem}  The number of ways of placing n similar objects into k different
  bins allowing empty bins is \(\binom{n + k - 1}{k - 1}
  \). The number of solutions to
  \(x_{1} + x_{2} + \ \ldots + x_{k} = n\) where each \(x_{i} \geq 0\) is
  the number of such distributions. Notice here that allowing, \(x_{2}\),
  for example, to be 0 is the same as saying that no objects will be
  placed in bin \#2.
\end{theorem}

Now let's turn to the first problem. Using the Principle of
Inclusion-Exclusion (PIE) we see that the number of functions from \{1,
2, 3, 4, 5\} \emph{onto} \{a, b, c\} is
\(3^{5} - 3 \bullet 2^{5} + 3 \bullet 1^{5}\). Similarly, the number of
functions from \{1, 2, 3, 4, 5, 6\} \emph{onto} \{a, b, c, d\} is

\(4^{6} - 4 \bullet 3^{6} + 6 \bullet 2^{6} - 4 \bullet 1^{6} =
\binom{4}{4}
4^{6} -
\binom{4}{3}
3^{6} +
\binom{4}{2}
2^{6} -
\binom{4}{1}
1^{6}.\) PIE can be used to show that, in general, the
number of functions from \(\{ x_{1},\ x_{2},\ \ldots,\ x_{n}\}\)
\emph{onto} \(\{ y_{1},\ y_{2},\ldots,\ y_{k}\}\) is

\[
\left( n,k \right) = k^{n} - \binom{k}{1} \left( k - 1 \right)^{n} + \binom{k}{2} \left( k - 2 \right)^{n} - \binom{k}{3} \left( k - 3 \right)^{n} + \ \ldots + \left( - 1 \right)^{k - 1}\binom{k}{k - 1} 1^{n}
\]

Each onto functions characterizes a distribution problem as follows: the
following onto function
\begin{longtable}[]{@{}llllll@{}}
\toprule
x & 1 & 2 & 3 & 4 & 5\tabularnewline
\midrule
\endhead
f(x) & a & a & b & c & b\tabularnewline
\bottomrule

\end{longtable}

can be considered as a model for distributing the two different objects
1 and 2 into a bin \emph{a} , the two different objects 3 and 5 into bin
\emph{b} and 4 into bin \emph{c} . Since each of the complete inverse
images \(f^{- 1}\left( a \right),\ f^{- 1}\left( b \right),\) and
\(f^{- 1}(c)\) is non-empty (the function being onto guarantees this)
this model counts distributions that do not allow empty bins. This
establishes the following distribution result.

\begin{theorem}
	 The number of ways of placing \emph{n} different objects into
\emph{k} different bins with no empty bins is \(B(n,k)\).
\end{theorem}
\begin{theorem}
	 The number of ways of placing n different objects into k
different bins in \(k^{n}.\)
\end{theorem}
Without the restriction of ``no empty bins'' this is just a matter of
saying that there are k places for disposing of each of the n objects.
This is also the number of functions from
\(\{ x_{1},\ x_{2},\ \ldots,\ x_{n}\}\) \emph{to}
\(\{ y_{1},\ y_{2},\ldots,\ y_{k}\}\), ignoring the restriction that
they must be onto.

Finally, we look at what happens if the n objects are different but the
k bins are now similar, still with no empty bins allowed. Since there
are k! ways to label the k bins, there are k! ways to convert similar
bins into different bins. The number of distributions of this type is
therefore \(\frac{1}{k!}B(n,k)\). For convenience of notation, let's
call this last expression: \(S\left( n,k \right).\)

\begin{theorem}
	 The number of ways of placing n different objects into k similar
bins with no empty bins is \(S\left( n,k \right).\)
\end{theorem}

There is one additional distribution problem that we can analyze using
the machinery developed thus far. In how many ways can you place n
different objects into k similar bins \emph{allowing} empty bins. Using
the sum rule we can look at the following disjoint cases:

No box is empty- \(S(n,\ \ k)\)

Exactly one box is empty-\(S(n,\ k - 1)\)
\begin{quote}
Exactly two bins are empty-\(S(n,\ k - 2)\)

\end{quote}

\(\vdots\)

All but one bin are empty-\(S(n,\ 1)\)

\begin{theorem} The number of ways of distributing n different objects into k or
fewer bins is
\(S\left( n,\ 1 \right) + S\left( n,\ 2 \right) + \ \ldots + S(n,\ k)\).
\end{theorem}

Finally, the theory of linear or integer \emph{partitions} solves the
final distribution problem: In how many ways can you distribute n
similar objects into k similar bins? Recall that a \emph{partition}
\(\left( a_{1},\ a_{2},\ \ldots,\ a_{k} \right)\ \)of an integer n is an
array of integers \(a_{i}\) such that
\(n = a_{1} + a_{2} + \ \ldots + a_{n}\) and
\(a_{1} \geq a_{2} \geq a_{3} \geq \ \ldots \geq a_{k} > 0\). For
example, the 5 partitions of n=4 are 4, 31, 22, 211, 1111. Here, the
distribution 31 is the same as 13 since the bins are similar.

DISTRIBUTIONS OF OBJECTS INTO BINS
\begin{longtable}[]{@{}lllll@{}}
\toprule
n objects & & & k bins &\tabularnewline
\midrule
\endhead
& & & Allow empty & \(P_{k}(n)\)\tabularnewline
S & (Partitions of Integers) & S & &\tabularnewline
& & & Do not & \(P_{k}(n - k)\)\tabularnewline
& & & Allow empty & \(\binom{n + k - 1}{k - 1} \)\tabularnewline
S & (Compositions) & D & &\tabularnewline
& & & Do not & \(\binom{n - 1}{k - 1} \)\tabularnewline
& & & Allow empty &
\(S\left( n,1 \right) + S\left( n,2 \right) + \ \ldots + S(n,k)\)\tabularnewline
D & (Set Partitions) & S & &\tabularnewline
& & & Do not & \(\frac{1}{k!}B(n,k)\)\tabularnewline
& & & Allow empty & \(k^{n}\)\tabularnewline
D & (Surjections) & D & &\tabularnewline
& & & Do not & \(B(n,k)\)\tabularnewline
\bottomrule

\end{longtable}

CONVENIENT NOTATION

S = Similar

D = Different

\(B\left( n,k \right) = k^{n} -
\binom{k}{1}
\left( k - 1 \right)^{n} +
\binom{k}{2}
\left( k - 2 \right)^{n} + \ \ldots + \left( - 1 \right)^{k - 1}
\binom{k}{k - 1}
1^{n}\)

\[
\left( n,k \right) = \frac{1}{k!}B\left( n,k \right)\
\]

COMBINATORIAL PROOFS

A combinatorial proof is sometimes used to show that two very different
looking expressions are in fact equal. The technique is as follows.
Refer to the two different looking expressions as the left-hand side
(LHS) and the right-hand side RHS. Create a situation or question that
is answered by the LHS; then show that the RHS also answers the
question. The conclusion is that LHS=RHS.

In the following we present a number of theorems, statements,
identities, etc, and give combinatorial proofs of each. For each result
the reader is urged to attempt an alternate proof for comparison
purposes. Such an alternate proof could be algebraic or geometric in
nature; or one could try a collapsing sum or an induction proof.

\emph{THEOREM 1} \(\binom{2n}{2}
 = 2
\binom{n}{2}
 + n^{2}\)

Split the \(2n\) objects into two groups A and B as shown:

\textbf{. . . . . . . . . . }

A B

First, you can choose 2 objects from a set of \(2n\ \)objects
in\(\
\binom{2n}{2}
\ \)ways. Alternatively, you could select two from group A
in\(\
\binom{n}{2}
\) ways or two from group B in\(\
\binom{n}{2}
\) ways or take one from each in \(n \bullet n = n^{2}\)
ways. Now add\(\
\binom{n}{2}
 +
\binom{n}{2}
 + n \bullet n\) and the result follows. The reader should
attempt an algebraic proof using the factorial formula
for\(\
\binom{n}{k}
\).

\emph{THEOREM 2} \(\binom{m + n}{2}
 -
\binom{m}{2}
 -
\binom{n}{2}
 = mn\)

Suppose you have a group of m men and n women and you want to form
men-women dancing pairs. This can clearly be done in mn ways. Or, you
could choose 2 from the total of m + n and delete the men-men pairs
(there are \(\binom{m}{2}
\ \)of these) and delete the women-women pairs
(also\(\binom{n}{2}
\)of these). The result follows. The reader could attempt
an algebraic proof or perhaps a geometric proof making use of figures
consisting of triangular numbers. Also, as a challenge, the reader could
formulate a similar result involving \(\binom{a + b + c}{3}
\) and a corresponding proof.

\emph{THEOREM 3} \(\binom{n}{0}
 +
\binom{n}{1}
 +
\binom{n}{2}
 + \ \ldots +
\binom{n}{n}
 = 2^{n}\)

Here again we first create a question that is answered by either side of
the given identity. Question: How many subsets does
\(\{ a_{1},\ a_{2},\ldots,\ a_{n}\}\) have? An n-set has
\(\binom{n}{2}
\) subsets. The left-hand side counts these subsets by
their size. There are \(\binom{n}{k}
\ \)subsets of size k.

In this situation the reader might \emph{not} want to try this
algebraically.

\emph{THEOREM 4} \(\binom{n}{1}
 + 2
\binom{n}{2}
 + 3
\binom{n}{3}
 + \ \ldots + n
\binom{n}{n}
 = {n2}^{n - 1}\)

Given a set of n people we can select a committee of size k along with a
chair from that committee in \(k
\binom{n}{k}
\ \)ways. We can select a committee (of size 1, or size 2,
or \ldots{}) and its chair in \(\binom{n}{1}
 + 2
\binom{n}{2}
 + 3
\binom{n}{3}
 + \ \ldots + n
\binom{n}{n}
\ \)ways. Alternatively, we can explain the term
\(n2^{n - 1}\ \)as follows: choose one of the n people to chair any of
the \(2^{n - 1}\)subsets of the remaining \(n - 1\) people.

The reader is invited to investigate one or more of the following
approaches: \(n2^{n - 1}\) looks like a derivative, so try
differentiating \(\left( 1 + x \right)^{n}\); a reverse and add approach
also works; or, first prove \(k
\binom{n}{k}
 = n
\binom{n - 1}{k - 1}
\ \)and then use it.

\emph{THEOREM 5} \(\
\binom{n}{k}
 =
\binom{n - 1}{k}
 +
\binom{n - 1}{k - 1}
\)

\(\binom{n}{k}
\ \)is the number of subsets of
\(\{ a_{1},\ a_{2},a_{3},\ \ldots,\ a_{n}\}\) of size k. Now a subset A
of size k either contains the fixed element\(\ a_{i}\) or it does not.
If A contains \(a_{i}\), the remaining \(k - 1\) elements can be
selected in \(\binom{n - 1}{k - 1}
\) ways. If, on the other hand, A does not contain
\(a_{i}\), you can choose the k elements from the depressed set
\(\left\{ a_{1},\ a_{2},\ \ldots,\ a_{i - 1,}a_{i + 1,\ \ldots,\ }a_{n} \right\}\ \)in
\(\binom{n - 1}{k}
\) ways. Since these two cases are mutually exclusive the
theorem follows.

Once again the reader is invited to attempt an algebraic proof.

\emph{THEOREM 6} Let \(d_{n}\) denote the number of derangements of 1,
2, 3, \ldots{}, n with

\(d_{0} = 1,\ d_{1} = 0.\) Then
\(d_{n} = \left( n - 1 \right)(d_{n - 1}{+ d}_{n - 2})\) for
\(n \geq 2\).

In forming a derangement of 1, 2, 3, \ldots{}, n the integer n can be
placed in any of the \(n - 1\) spots 1, 2, 3, \ldots{}, \(n - 1\), say
spot i. If i goes into spot n there are \(d_{n - 2}\) ways to finish it.
If i does not go into spot n there are \(d_{n - 1}\) ways to complete
the derangement.

The reader can use the principle of inclusion-exclusion to derive a
formula for \(d_{n}\)

from which the new recursion
\(d_{n} = nd_{n - 1} + \left( - 1 \right)^{n}\), and the above
recursion, can be derived.

\emph{THEOREM 7} \(\binom{n}{0}
^{2} +
\binom{n}{1}
^{2} +
\binom{n}{2}
^{2} + \ \ldots +
\binom{n}{n}
^{2} =
\binom{2n}{n}
\)

Given a group of \(2n\) people consisting of n men and n women, in how
many ways can one choose a group of n people? The answer to that
question is just\(\binom{2n}{n}
\), the right side of the identity in question. One could
also form the group of n people in the following way: choose 0 men and n
women in

\(\binom{n}{0}
\binom{n}{n}
 =
\binom{n}{0}
^{2}\ \)ways, or, choose 1 man and \(n - 1\) women in

\(\binom{n}{1}
\binom{n}{n - 1}
 =
\binom{n}{1}
^{2}\)ways, or choose 2 men and \(n - 2\) women in

\(\binom{n}{2}
\binom{n}{n - 2}
 =
\binom{n}{2}
^{2}\) ways and so on. Now add these disjoint cases.

An alternate algebraic proof is less interesting: Extract the
coefficient of \(x^{n}\) from both sides
of.\(\left\lbrack \left( x + 1 \right)^{n} \right\rbrack^{2} = \left( x + 1 \right)^{2n}\)

\emph{THEOREM 8} \(\binom{2}{2}
 +
\binom{3}{2}
 +
\binom{4}{2}
 + \ \ldots +
\binom{n}{2}
 =
\binom{n + 1}{3}
\)

The term \(\binom{n + 1}{3}
\) is the number of binary strings of length \(n + 1\)
consisting of three 1's (and the rest 0's). The left hand side counts
these by where in the string the left- most 1 appears. Let
\(a_{1}\ a_{2}a_{3}\ldots a_{n + 1}\) be a string of length \(n + 1\).
There are \(\binom{n}{2}
\) strings when \(a_{1} = 1,\
\binom{n - 1}{2}
\) strings when \(a_{2} = 1\) is the leftmost 1, \ldots{},
and \(\binom{2}{2}
\)strings when \(a_{n - 1} = 1\ \)is the leftmost 1. In
this last case the string looks like 000 \ldots{} 0111.

Attempting a proof by mathematical induction is an easy option. An
algebraic approach is not!

\emph{THEOREM 9} The number of positive integers that have their digits
in strictly increasing order is \(2^{9} - 1\). Include single digit
numbers.

There are \(\binom{9}{1}
\) single digit type, \(\binom{9}{2}
\) double digit type (just select 2 of the 9 digits 1, 2,
3, \ldots{}, 9 and arrange in order), \ldots{}, and so on to see that
there are \(\binom{9}{9}
\) nine digit type. The total is \(\binom{9}{1}
 +
\binom{9}{2}
 +
\binom{9}{3}
 + \ \ldots +
\binom{9}{9}
 = 2^{9} - 1.\)

Here is an alternative, more clever, proof. Look at 123456789. Any
increasing number can be made by deleting \emph{any} subset of digits,
except all of them. There are \(2^{9} - 1\)such subsets. For example,
delete the subset \{2, 4, 7, 9\} and you get 13568. Combining these two
approaches actually gives you a nice proof that \(\binom{9}{0}
 +
\binom{9}{1}
 +
\binom{9}{2}
 + \ \ldots +
\binom{9}{9}
 = 2^{9}\).

\emph{THEOREM 10} \(\binom{3n}{3}
 = 3
\binom{n}{3}
 + 6n
\binom{n}{2}
 + n^{3}\)

This one is a little tougher. First rewrite as \(n^{3} =
\binom{3n}{3}
 - 3
\binom{n}{3}
 - 6n
\binom{n}{2}
\). Suppose you have n men, n women and n children and you
want to select triples consisting of one man, one woman and one child.
There are \(n^{3}\) ways to do this, just pick one from each group.
Alternatively, select 3 of the 3n people in \(\binom{3n}{3}
\) ways and delete the ``bad'' ones. Delete the ones where
you selected all three from one group -- there are \(3
\binom{n}{3}
\)of these. Now delete those where you had two from one
group and one from another -- there are \(2n
\binom{n}{2}
 + 2n
\binom{n}{2}
 + 2n
\binom{n}{2}
\) of these.

\emph{THEOREM 11}
\(1 \bullet 1! + 2 \bullet 2! + 3 \bullet 3! + \ \ldots + n \bullet n! = \left( n + 1 \right)! - 1\)

In how many ways can you arrange the n+1 numbers 0, 1, 2, \ldots{}, n so
that they are \emph{not} in ascending order? The answer is
\(\left( n + 1 \right)! - 1\) since 0, 1, 2, \ldots{}, n is the
\emph{only} arrangement in ascending order. Now lets separate into
cases. Let \(a_{0},\ a_{1},\ a_{2},\ \ldots,\ a_{n}\ \)represent an
arrangement of these n+1 numbers. If \(a_{0} \neq 0\), there are n
choices left for \(a_{0}\), and then n! ways to fill out
\(a_{1},\ a_{2},\ \ldots,\ a_{n}\) for a total of \(n \bullet n!\). Now
let \(a_{0} = 0\) but \(a_{1} \neq 1\). There are \(n - 1\) choices for
\(a_{1}\) and \(\left( n - 1 \right)!\ \)ways to complete for a total of
\(\left( n - 1 \right)\left( n - 1 \right)!\). Now continue with
\(a_{0} = 1,\ a_{1} = 1\) but \(a_{2} \neq 2\). There are
\(\left( n - 2 \right)\left( n - 2 \right)!\) ways, and so on.

The reader should attempt a collapsing sum or induction proof.

\emph{THEOREM 12}
\(1 \bullet n + 2\left( n - 1 \right) + 3\left( n - 2 \right) + \ \ldots + n \bullet 1 =
\binom{n + 2}{3}
\)

Let \(S = \{ 0,\ 1,\ 2,\ \ldots,\ n,\ n + 1\}\). The number of subsets
of S of size 3 is \(\binom{n + 2}{3}
\). Each one looks like \{a, b, c\} with a \textless{} b
\textless{} c, Let's count these by looking at the size of the middle
element b. If b=1 , there is one choice for a, namely a=0 and n choices
for c for a total of \(1 \bullet n\). If b=2 there are 2 choices for a,
and \(n - 1\) choices for c for a total of \(2(n - 1)\). If b=3 the
total is

\(3(n - 2)\), and so on. The total derived by looking at cases is

\(1 \bullet n + 2\left( n - 1 \right) + 3\left( n - 2 \right) + \ \ldots + n \bullet 1\)
and this must equal \(\binom{n + 2}{3}
\) since the cases are disjoint.

\emph{\\
}

\emph{THEOREM 13} \(k
\binom{n}{k}
 = n
\binom{n - 1}{k - 1}
\)

Suppose you have a group of n people and you wish to form a subcommittee
of k people with one of those k people to serve as chair. Choose the
subcommittee in \(\binom{n}{k}
\) ways and the chair in k ways. The product rule
gives\(\text{\ k}
\binom{n}{k}
\) as the number of ways of selecting such a chaired
subcommittee.

Alternatively, you could first choose any one of the n people to serve
as chair and then fill out the committee in \(\binom{n - 1}{k - 1}
\) ways. There are \(n
\binom{n - 1}{k - 1}
\) ways to select a chaired subcommittee. Hence
\(k
\binom{n}{k}
 = n
\binom{n - 1}{k - 1}
\)

The reader should attempt the easier algebraic technique by converting
\(\binom{n}{k}
\) to factorial form.

\emph{THEOREM 14} \(P\left( n,k \right) = k!
\binom{n}{k}
\)

Questions -- How many permutations are there of k objects chosen from a
collection of n objects? The LHS answers the question. There are
\(P\left( n,k \right) = n\left( n - 1 \right)\left( n - 2 \right)...(n - k + 1)\)
ways. Alternatively, one could first choose the k objects from the n
objects in \(\binom{n}{k}
\) ways and then permute these in k! ways.

\emph{\\
THEOREM 15} \(n2^{n - 1} = 1
\binom{n}{1}
 + 2
\binom{n}{2}
 + 3
\binom{n}{3}
 + \ \ldots + n
\binom{n}{n}
\)

Contrast this discussion with that presented in THEOREM 4. Look at the
set of the first \(2^{n}\) nonnegative
integers\(\ 0,\ 1,\ 2,\ \ldots,\ 2^{n} - 1\). When you convert each to
binary form what is the total number of 1s written? This binary list
will look like the standard listing in \(B^{n}\) the set of all binary
strings of length n. For n=3
\(B^{3} = \{ 000,\ 001,\ 010,\ 011,\ 100,\ 101,\ 110,\ 111\}\). In
\(B^{n}\) each string has length n and there are \(2^{n}\) of them. But
half of the \(n \bullet 2^{n}\) symbols are 1's. Then the total number
of 1's is \(n2^{n - 1}\). Alternatively, we could consider each string
and count those with one 1, then those with two 1's, etc. There are
\(1 \bullet
\binom{n}{1}
\) with one 1, \(2
\binom{n}{2}
\) total 1's in those binary numbers with exactly two 1's,
\(3
\binom{n}{3}
\) in those with exactly three 1's , and so on. The total
is \(1
\binom{n}{1}
 + 2
\binom{n}{2}
 + 3
\binom{n}{3}
 + \ \ldots + n
\binom{n}{n}
\text{.\ }\)The result now follows by equating
\(n2^{n - 1}\) to this sum.

\emph{THEOREM 16} \(\binom{n}{0}
^{2} +
\binom{n}{1}
^{2} +
\binom{n}{2}
^{2} + \ \ldots +
\binom{n}{n}
^{2} =
\binom{2n}{n}
\)

Let's revisit this identity using equivalence relations. A binary
relation R on the set of all binary strings of length n is defined by
specifying that \(\left( \alpha,\ \beta \right) \in R\) whenever weight
\(\alpha =\) weight \(\beta\). This R is an equivalence relation. For
n=3 there are four different equivalence classes, each containing
strings of weight 0, 1, 2, or 3. The relation R contains
\(1^{2} + 3^{2} + 3^{2} + 1^{2}\) ordered pairs; for example, with
weight 1 each of the three strings 001, 010, 100 can be paired with any
one of those same strings for a total of \(3^{2} = 9\). These ordered
pairs can be counted in another way. Each ordered pair looks like
\(\left( - - - ,\  - - - \right).\) Place 1's in any three positions,
and 0's in the others. If you take the complement of the entries in the
second coordinate an element of R is produced. Here is what one sequence
of this process looks like:

\(\left( - - - , - - - \right) \rightarrow \left( - 11,\  - - 1 \right) \rightarrow \left( 011,\ 001 \right) \rightarrow (011,\ 110)\).
The reader can check that this process always produces an element of R
and that the case of n=3 extends easily to general n. Conclusion: choose
the n positions for 1's in \(\binom{2n}{n}
\) ways. The result follows.

\emph{THEOREM 17} \(\binom{n}{0}
d_{0} +
\binom{n}{1}
d_{1} +
\binom{n}{2}
d_{2} + \ \ldots +
\binom{n}{n}
d_{n} = n!\) where \(d_{n}\) denotes the
\(\ n^{\text{th}}\ \)derangement number, \(d_{0} = 1,\ d_{1} = 0.\)

The right-hand side, n!, gives the number of permutations of n objects.
So the left-hand side must provide the same enumeration. The left side
partitions the permutations according to how many elements are deranged
(and the rest fixed). The term \(\binom{n}{i}
d_{i} =
\binom{n}{n - i}
d_{\text{i\ }}\)gives the number of permutations of n where
\(n - i\) elements are fixed and the remaining i elements are deranged.
Summing over all i yields

\(\binom{n}{n}
d_{0} +
\binom{n}{n - 1}
d_{1} + \ \ldots +
\binom{n}{0}
d_{n} =
\binom{n}{0}
d_{0} +
\binom{n}{1}
d_{1} + \ \ldots +
\binom{n}{n}
d_{n} = n!\)

\emph{THEOREM 18} \(F_{n + 1} =
\binom{n}{0}
 +
\binom{n - 1}{1}
 +
\binom{n - 2}{2}
 + \ \ldots\) where \(F_{n}\) denotes the
\(\ n^{\text{th}}\ \)Fibonacci number.

Here is a question that might resolve the issue. How many different
brick paths of length n (and width 1) can you make using 1 x 1 bricks
and 1 x 2 bricks? Let a(n) denote the number of such paths of length n.
A few drawings will show that a(1)=1, a(2)=2, a(3)=3, a(4)=5. Since you
can place a 1 x 1 brick in front of all paths of length \(n - 1\) or a 1
x 2 brick in front of all paths of length \(n - 2\) we have that
\(a\left( n \right) = a\left( n - 1 \right) + a\left( n - 2 \right)\).
This recursion, along with the initial conditions, show that
\(a\left( n \right) = F_{n + 1}\), the left-hand side of the identity.

Now lets look at all paths of length n and count them by the number of 1
x 2 bricks. If there are i 1 x 2 bricks there are \(n - i\) total bricks
making up the path of length n. Choose the positions of the i 1 x 2
bricks in \(\binom{n - i}{i}
\) ways. Now sum as i ranges through the values

0, 1, 2, \ldots{}, and obtain \(\binom{n}{0}
 +
\binom{n - 1}{1}
 +
\binom{n - 2}{2}
 + \ \ldots = F_{n + 1}\).

The reader should draw all paths of length \(n = 5\), for example, and
examine the cases with i=0, 1, 2 . The reader could also explore other
proofs.

\emph{THEOREM 19} \(\binom{m + n}{2}
 -
\binom{m}{2}
 -
\binom{n}{2}
 = mn\) (Revisited)

1.

m

n
\begin{figure}[h!]\begin{center}
\includegraphics[width=0.70\columnwidth]{figures/image16/default-figure}
\caption{{Couldn't find a caption, edit here to supply one.%
}}
\end{center}
\end{figure}

2.
There are mn one by one squares in the subdivided

m by n rectangle. Each choice of arrows

(one horizontal, one vertical) specifies one of these squares.

Pick two arrows but don't take two from the top or two from

the side. Then \(\binom{m + n}{2}
 -
\binom{m}{2}
 -
\binom{n}{2}
 = mn\)

3. Take m people in one group and n in another. How many handshakes can
be accomplished? Among the m people there are \(\binom{m}{2}
\) handshakes; among the n people there are
\(\binom{n}{2}
\ \)handshakes and between the two groups, mn. But,
\(\binom{m + n}{2}
\) also represents the total number of handshakes among the
people. Then we get: \(\binom{m}{2}
 +
\binom{n}{2}
 + mn =
\binom{m + n}{2}
\).
\begin{figure}[h!]\begin{center}
\includegraphics[width=0.70\columnwidth]{figures/image18/image18}
\caption{{Couldn't find a caption, edit here to supply one.%
}}
\end{center}
\end{figure}


\end{document}

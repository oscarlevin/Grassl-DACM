% This file was converted to LaTeX by Writer2LaTeX ver. 1.4
% see http://writer2latex.sourceforge.net for more info
\documentclass{article}
\usepackage[utf8]{inputenc}
\usepackage[LGR,T1]{fontenc}
\usepackage[greek,english]{babel}
\usepackage{amsmath}
\usepackage{amssymb,amsfonts,textcomp}
\usepackage{array}
\usepackage{supertabular}
\usepackage{hhline}
\usepackage{graphicx}
\makeatletter
\newcommand\arraybslash{\let\\\@arraycr}
\makeatother
\setlength\tabcolsep{1mm}
\renewcommand\arraystretch{1.3}
\title{}
\author{bernadette.mendozasp}
\date{2009-08-06}
\begin{document}
Discrete and Combinatorial Mathematics

MATH 528

[Warning: Draw object ignored]

Richard Grassl 

University of Northern Colorado

School of Mathematical Sciences

Richard.grassl@unco.edu

Partially Funded by NSF Grant \#0832026

Revised August 2009

Table of Contents 

A Brief History of Fibonacci Numbers………………………………….….…1

The Catalan Numbers………………………………………………………….7

Stirling Numbers………………………………………………………….….18

The Bell Numbers……………………………………………………...….. .. 24

Group Projects………………………………………………………….…….33

\ \ Paths and Binomial Coefficients…………………………….…….…..34

\ \ Triangular Numbers ………………………………………..…….…....35

\ \ Fibonacci Numbers…………………………………………………….36

\ \ Generalized Pascal Triangles……………………………………..……37

\ \ A Pascal-like Triangle of Eulerian Numbers……………………..……38

\ \ Leibnitz Harmonic Triangle………………………………………..…..39

\ \ The 12 Days of Christmas…………………………………………..….40

Four (or Eight) Distribution Problems…………………………………….......41

Nineteen Combinatorial Proofs…………………………………………… .....45

Linear Partitions………………………………………………………………1-8

The Exponential Generating Functions………………………………………1-11

A BRIEF HISTORY OF FIBONACCI NUMBERS

\ \ Fibonacci numbers receive their name from Leonardo of Pisa (Leonardo Pisano, c. 1175-1250), better known as Leonardo
Fibonacci. Fibonacci is a contraction of Filius Bonacci, son of Bonacci.

\ \ Leonardo was born about 1175 in the commercial center of Pisa. This was a time of great interest and importance in
the history of Western Civilization. One finds the influence of the crusades stirring and awakening the people of
Europe by bringing them in contact with the more advanced intellect of the East. During this time the Universities of
Naples, Padua, Paris, Oxford, and Cambridge were established, the Magna Carta signed in England, and the long struggle
between the Papacy and the Empire was culminated. Commerce was flourishing in the Mediterranean world and adventurous
travelers such as Marco Polo were penetrating far beyond the borders of the known world.

\ \ It is in this growing commercial activity that we find the young Leonardo at Bugia on the Northern coast of Africa.
Here the merchants of Pisa and other commercial cities of Italy had large warehouses for the storage of their goods.
Actually very little is known about the life of this great mathematician. No contemporary historian makes mention of
him, and one must look to his writings to find information about him.

\ \ A mathematician before his time, Leonardo of Pisa, alias Leonardo Pisano, alias Leonardo Bigollo, alias Fibonacci,
was a despair to his teacher as a young boy and an enigma to his colleagues in his later years. Convinced of the
superiority of the Hindu-Arabic numeral system over the Roman system, Leonardo wrote one of his greatest works, Liber
Abaci (in English, “Book of Calculating”) to introduce this system to the Western world. The Liber Abaci was written in
1202 but was not published until 1857 because “it was too advanced for Leonardo’s contemporaries”. Along with the
introduction and development of many mathematical topics, the Liber Abaci contained interesting story problems that
Leonardo liked to invent. His most popular problem is the breeding pair of rabbits: “How many pairs of rabbits will
there be after a year if it is assumed that every month each pair produces one new pair, which begins to bear young two
months after its own birth?”

\ \ This problem generates the infinite sequence that bears his name because his work is the earliest known recording.
The Fibonacci sequence begins with 1 and each number that follows is the sum of the previous two numbers. The first ten
Fibonnaci numbers are: 1, 1, 2, 3, 5, 8, 13, 21, 34, 55.

\ \ Leonardo, in 1228, gave a second edition of the Liber Abaci which he dedicated to Michel Scott, astrologer to the
Emperor Frederic II and author of many scientific works. Copies of this edition exist today. Leonardo profusely
illustrated and strongly advocated the Hindu-Arabic system in this work. He gave an extensive discussion of the Rule of
False Position and the Rule of Three. Leonardo did not use a general method in problem solving; each problem was solved
independently of the others. In the solution of a problem he not only considered the problem as it might occur, but
considered all of the variations of the question, even those that were not reasonable. 

\ \ Because of Leonardo’s great reputation, the emperor Frederick II, when in Pisa (1225), held a sort of mathematical
tournament to test Leonardo’s skill. The competitors were informed beforehand of the questions to be asked, some or all
of which were composed by Johannes of Palermo, who was of Frederick’s staff. This is the first case in the history of
mathematics that one meets with an instance of these challenges to solve particular problems which were so common in
the sixteenth and seventeenth centuries.

\ \ The first question propounded was to find a number of which the square when decreased or increased by 5 would remain
a square. The correct answer given by Leonardo was 41/12. The next question was to find by the methods used in the
tenth book of Euclid a line whose length x should satisfy the equation x³ + 2x² + 10x – 20 = 0. Leonardo showed by
geometry that the problem was impossible, but he gave an approximation of the root 1.3688081075…, which is correct to
nine places. Liber Abaci contains many additional problems of this type.

\ \ After the ‘Rabbit Problem’, the matter lay for 400 years. In 1611, Johann Kepler, of astronomy fame, arrived at the
series 1, 1, 2, 3, 5, 8, 13, 21, … . There is no indication that he had access to one of Fibonacci’s hand-written books
(The Liber Abaci was not published until 1857).

\ \ Simon Stevens (1548-1620) wrote on the famous Golden Section. The editor of his works, A. Gerard arrived at the
following formula for expressing the series in 1634:  $F_{n+2}=F_{n+1}+F_n$

\ \ A hundred years must pass before the problem is again considered. In 1753, R. Simpson derived a formula, implied by
Kepler

\begin{equation*}
F_{n-1}F_{n+1}-F_n^2=\left(-1\right)^n
\end{equation*}
\ \ A second hundred years pass by and the series again comes under study. In 1843, J.P.M. Binet derives an analytical
function for determining the value of any Fibonacci number

\section[2n Fn = (1 +)n {}- (1 {}- )n]{2n $\sqrt 5$ Fn = (1 + $\sqrt 5$)n - (1 -  $\sqrt 5$)n}
In 1846 E. Catalan derived the formula:

\section[]{ $2^{n-1}F_n=\frac n
1+\frac{5n\left(n-1\right)\left(n-2\right)}{1{\bullet}2{\bullet}3}+\frac{5^2n\left(n-1\right)\left(n-2\right)\left(n-3\right)\left(n-4\right)}{1{\bullet}2{\bullet}3{\bullet}4{\bullet}5}$
}
\ \ By now, the series had received enough attention to deserve a name. It was variously called the Braun series, the
Schimper-Braun series, the Lamé series and the Gerhardt series. A. Braun, applied the series to the arrangement of the
scales of pine cones. Schimper is completely unknown. Gerhardt is probably a misspelling of Girard. 

\ \ Edouard Lucas, who dominated the field of recursive series during the period 1876-1891, first applied Fibonacci’s
name to the series and it has been known as the Fibonacci series since then. 

\ \ About this time, 1858, Sam Loyd claimed to have invented the checkerboard paradox. It is first found in print in a
German journal in 1868. Today it seems proper to call it the Carroll Paradox after Lewis Carroll (Charles Dodgson,
1832-1893) who was quite fond of it.

\section[PROBLEMS ON FIBONACCI NUMBERS]{PROBLEMS ON FIBONACCI NUMBERS}
\begin{enumerate}
\item Determine a formula for each:
\end{enumerate}
\begin{enumerate}
\item  $F_0+F_1+F_2+{\dots}+F_m$\ \ \ \ (b)  $F_0+F_2+F_4+{\dots}+F_m$  where m is even.
\end{enumerate}
\begin{enumerate}
\item Prove each of the following
\end{enumerate}
\begin{enumerate}
\item  $F_{n+1}^2-F_n^2=F_{n-1}F_{n+2}$\ \ \ \ (b)  $F_k^2=F_k(F_{k+1}-F_{k-1})$
\end{enumerate}
\begin{enumerate}
\item Determine a closed expression for F0F3 + F1F4 + … + Fn-1Fn+2 using 2(a).
\item Determine a formula for  $F_1^2+F_2^2+{\dots}+F_n^2$ in two ways:
\end{enumerate}
\begin{enumerate}
\item Collect data and prove by mathematical induction.
\item Use 2(b) and telescoping sums. 
\end{enumerate}
\begin{enumerate}
\item Give a geometrical “proof” for the result in Problem 4.
\item Prove 
$\left(\begin{matrix}0&1\\1&1\end{matrix}\right)^n=\left(\begin{matrix}F_{n-1}&F_n\\F_n&F_{n+1}\end{matrix}\right)$ by
induction. For convenience let Q = $\left(\begin{matrix}0&1\\1&1\end{matrix}\right)$.
\item Prove that  $F_{n-1}F_{n+1}-F_n^2=(-1)^n$ in two ways:
\end{enumerate}
\begin{enumerate}
\item Mathematical induction.
\item Use Problem 6 and determinants.
\end{enumerate}
\begin{enumerate}
\item Is there a result analogous to that in Problem 7 for just the integers 1, 2, 3, 4, …?
\item Show that  $Q^{2n+1}=Q^nQ^n$+1  and that  $F_{2n+1}=F_{n+1}^2+F_n^2$.
\item Prove the Binet Formula:  $F_{n=?\frac{a^n-b^n}{a-b}}$ where a, b are roots of x2 - x - 1 = 0.
\item Prove 
$\left(\genfrac{}{}{0pt}{0}n0\right)F_0+\left(\genfrac{}{}{0pt}{0}n1\right)F_1+{\dots}+\left(\genfrac{}{}{0pt}{0}nn\right)F_n=F_{2n}$
in two ways:
\end{enumerate}
\begin{enumerate}
\item Use the Binet Formula.
\item Use Q.
\end{enumerate}
\begin{enumerate}
\item Prove that  $\sum _{n=2}^{{\infty}}\frac 1{F_{n-1}F_{n+1}}=1.$
\end{enumerate}
 Hint: Show first that   $\frac 1{F_{n-1}F_{n+1}}=\frac 1{F_{n-1}F_n}-\frac 1{F_nF_{n+1}}$

\begin{enumerate}
\item Prove that  $\sum _{n=2}^{{\infty}}\frac{F_n}{F_{n-1}F_{n+1}}=2.$
\item Prove that  $\lim _{n\rightarrow {\infty}}\frac{F_{n+1}}{F_n}=\frac{1+\sqrt 5} 2.$
\item For which values of n is Fn an integral multiple of 3?
\item Can you have four distinct positive Fibonacci numbers in arithmetic progression?
\item How many Fibonacci numbers are perfect squares?
\item Find a closed formula for  $\frac 1{F_1}+\frac 1{F_2}+\frac 1{F_4}+\frac 1{F_8}+{\dots}.$
\item Conjecture and prove a formula for 
$\left(\genfrac{}{}{0pt}{0}n0\right)+\left(\genfrac{}{}{0pt}{0}{n-1}1\right)+\left(\genfrac{}{}{0pt}{0}{n-2}2\right)+{\dots}.$
\item In what sense is  $\frac x{1-x-x^2}$ the generating function for the sequence of Fibonacci numbers?
\item Prove that  $F_{5n+5}=3F_{5n}+5F_{5n+1}$  Hint: Use  $Q^{5n+5}$.
\item Let \textgreek{f }be the positive root of x2 - x - 1 = 0. Prove that  $\varphi ^n=F_n\varphi +F_{n-1}$.
\item Can every positive integer be written as a sum of distinct Fibonacci numbers?
\end{enumerate}
THE CATALAN NUMBERS

 (Notes of a talk presented by Professor Richard Grassl, Muhlenberg College, March 15, 1991)

The Sequence of Catalan numbers, named after Eugene Catalan who along with Euler discovered many of the properties of
these numbers, is 1, 1, 2, 5, 14, 42, 132, … . The characterizations below show that a closed formula for these numbers
is given by  $C_{n+1}=C_1C_n+C_2C_{n-1}+{\dots}+C_nC_1$  with  $C_1=1,C_2=1,{\dots}$

\begin{enumerate}
\item Parenthesizing: In how many ways can you insert parentheses in the nonassociative product abcd? The five ways are
displayed
\end{enumerate}
(ab)(cd)  ((ab)c)d  a(b(cd))  (a(bc))d  a((bc)d)

There are two ways for abc, namely a(bc) and (ab)c, and one for (ab). For a product of n symbols a1a2…an, break it at
the k-th symbol:

\begin{equation*}
\left(a_1a_2a_3{\dots}a_k\right)(a_{k+1}{\dots}a_n)
\end{equation*}
Let Pn denote the number of ways of inserting integers into this product. There are Pk ways of inserting parentheses in 
$\left(a_1{\dots}a_k\right)$ and  $P_{n-k}$ ways for the factor  $(a_{k+1}{\dots}a_n)$. Letting k range from 1 to 
$n-1$  we have that

\begin{equation*}
P_n=\sum _{k=1}^{n-1}P_kP_{n-k}=P_1P_{n-1}+P_2P_{n-2}+{\dots}+P_{n-1}P_1
\end{equation*}
\begin{enumerate}
\item The H-D sequences: 2n people stand in line at a theatre. Admission is 50¢, (denoted by H), and the box office
starts with no change. n of the people have H and n have \$1(D). In how many ways can the 2n people line up so that all
can be admitted? Here we enumerate the number of workable sequences of n H’s and n D’s such that at each point in the
sequence the number of H’s is not less than the number of D’s.
\end{enumerate}
[Warning: Draw object ignored]The total number of sequences of n H’s and n D’s is 
$\left(\genfrac{}{}{0pt}{0}{2n}n\right)$. We delete the number of nonworkable sequences. Each such sequence has a first
snag as shown by the arrow in 

\section[H H D D D H H D D D H H]{H H D D D H H D D D H H}
\section[Reverse each letter up to and including the snag, obtaining]{Reverse each letter up to and including the snag,
obtaining}
\section[D D H H H H H D D D H H]{[Warning: Draw object ignored]D D H H H H H D D D H H}
a sequence of n+1 H’s and  $n-1$ D’s. For  $n=3$, H D D D H H is a nonworkable sequence with the snag indicated. Its
mate, found by reversing the first three letters, is 

D H H D H H, a sequence of 2D’s and 4H’s. Any arrangement of 2D’s and 4H’s will correspond to exactly one nonworkable
sequence; simply scan through and see the first time the H’s dominate the D’s and then reverse through that spot. There
are  $\left(\begin{matrix}6\\3\end{matrix}\right)=20$ arrangements of 3H’s and 3D’s. There are 
$\left(\begin{matrix}6\\2\end{matrix}\right)=15$ arrangements of 2D’s and 4H’s each of which corresponds to a
nonworkable sequence. Hence there are 
$\left(\begin{matrix}6\\3\end{matrix}\right)-\left(\begin{matrix}6\\2\end{matrix}\right)=5$ workable sequences. In
general,  $\left(\genfrac{}{}{0pt}{0}{2n}n\right)-\left(\genfrac{}{}{0pt}{0}{2n}{n-1}\right)=\frac
1{n+1}\left(\genfrac{}{}{0pt}{0}{2n}n\right)$ gives the number of workable sequences. See FIGURE 1.

For n = 3, the matching of the 5 workable sequences with one of the 5 ways of parenthesizing, given by the bijection
($\leftrightarrow $H, letter $\leftrightarrow $D is presented:

H H H D D D $\leftrightarrow $ (((abcd

H H D H D D $\leftrightarrow $ ((a(bcd

H H D D H D $\leftrightarrow $ ((ab(cd

H D H H D D $\leftrightarrow $ (a((bcd

H D H D H D $\leftrightarrow $(a(b(cd

\begin{enumerate}
\item [Warning: Draw object ignored]Paths:  A point (a, b) in the plane is a lattice point if a and b are integers. How
many paths of length 2n, consisting of horizontal and vertical segments of unit length, are there from (0, 0) to (n, n)
such that the path never goes above the line y = x? One such path to (3, 3) is shown.
\end{enumerate}
\begin{flushleft}
\tablefirsthead{}
\tablehead{}
\tabletail{}
\tablelasttail{}
\begin{supertabular}{|m{0.37055984in}|m{0.35875985in}|m{0.35875985in}|m{0.35875985in}m{0.35875985in}}
 &
 &
[Warning: Draw object ignored] &
(3,3) &
\\\hhline{----~}
 &
 &
 &
 &
\\\hhline{----~}
 &
 &
 &
 &
\\\hhline{----~}
[Warning: Draw object ignored] &
 &
 &
 &
\\\hhline{----~}
\end{supertabular}
\end{flushleft}
Using R for right and U for up, the sequence R U R R U U gives the path. The bijection R$\leftrightarrow $H and
U$\leftrightarrow $D links paths to H-D sequences.

\begin{enumerate}
\item [Warning: Draw object ignored]Triangulations of convex n{}-gons: Fix n vertices; in how many ways can diagonals be
inserted so as to decompose the n{}-gon into triangles. For n = 5  the five figures are
\end{enumerate}
\section[There are two drawings for n = 4]{[Warning: Draw object ignored]There are two drawings for n = 4}
\section{}
[Warning: Draw object ignored]A bijection between triangulations and parenthesizing is illustrated next:

\begin{enumerate}
\item Tableau Insertion: Insert the integers 1, 2, …, 2n into a 2 by n rectangle of boxes such that the entries are
monotonic in rows and columns. For n = 3 there are five arrangements:
\end{enumerate}
\begin{flushleft}
\tablefirsthead{}
\tablehead{}
\tabletail{}
\tablelasttail{}
\begin{supertabular}{|m{0.15385985in}|m{0.15455985in}|m{0.15455985in}|m{0.085159846in}|m{0.15455985in}|m{0.15455985in}|m{0.15455985in}|m{0.075459845in}|m{0.15455985in}|m{0.15455985in}|m{0.15455985in}|m{0.075459845in}|m{0.15455985in}|m{0.15455985in}|m{0.15455985in}|m{0.075459845in}|m{0.15455985in}|m{0.15455985in}|m{0.15665984in}|}
\hhline{---~---~---~---~---}
\centering 1 &
\centering 2 &
\centering 3 &
 &
\centering 1 &
\centering 2 &
\centering 4 &
 &
\centering 1 &
\centering 2 &
\centering 5 &
 &
\centering 1 &
\centering 3 &
\centering 4 &
 &
\centering 1 &
\centering 3 &
\centering\arraybslash 5\\\hline
\centering 4 &
\centering 5 &
\centering 6 &
 &
\centering 3 &
\centering 5 &
\centering 6 &
 &
\centering 3 &
\centering 4 &
\centering 6 &
 &
\centering 2 &
\centering 5 &
\centering 6 &
 &
\centering 2 &
\centering 4 &
\centering\arraybslash 6\\\hhline{---~---~---~---~---}
\end{supertabular}
\end{flushleft}
A bijection to the path problem is: an entry in the top row $\leftrightarrow $ R. For example, the middle arrangement
corresponds to R R U U R U.

\begin{enumerate}
\item Trivalent rooted trees: A tree is a connected graph that has no cycles. Trivalent means that each interior vertex
has degree 3. The “leaves” or endpoints have degree 1. The number of trivalent rooted trees having n-vertices (not
counting the root) is Cn. The cases n = 1, 2, 3, 4 are drawn in FIGURE 2 along with bijections linking H-D sequences
and parenthesizing. For the bijection, label interior vertices with a 1 and leaves with a 0. Start at the root, keep
bearing to the right, and call off the sequence of 0’s and 1’s, not repeating them once called.
\end{enumerate}
[Warning: Draw object ignored]A link between a particular triangulation of a 5-gon and its associated tree is displayed
in the following diagram:

[Warning: Draw object ignored]

Place a vertex in each triangle, connect the three vertices and run leaves out all edges, including the bottom which
forms the root. Straighten the tree out! FIGURE 3 shows the five triangulations of a convex pentagon along with their
trees.

\begin{enumerate}
\item Rhyme schemes for n line stanzas
\end{enumerate}
A 2-line poem can have just two rhyme schemes

\begin{equation*}
\begin{matrix}a\\a\end{matrix}\vee \begin{matrix}a\\b\end{matrix}
\end{equation*}
A 3-line poem can have five schemes:

\begin{equation*}
\begin{matrix}a\\a\\a\end{matrix}\begin{matrix}a\\a\\b\end{matrix}\begin{matrix}a\\b\\a\end{matrix}\begin{matrix}a\\b\\b\end{matrix}\begin{matrix}a\\b\\c\end{matrix}
\end{equation*}
The last pattern indicates that no line rhymes with any other. The Catalan numbers seem to be popping up again. For n =
4 there happen to be 15 (not 14) patterns:

\begin{flushleft}
\tablefirsthead{}
\tablehead{}
\tabletail{}
\tablelasttail{}
\begin{supertabular}{m{0.34275985in}m{0.34345984in}m{0.34275985in}m{0.34345984in}m{0.34275985in}m{0.34345984in}m{0.34275985in}m{0.34345984in}m{0.34275985in}m{0.34345984in}m{0.30805984in}m{0.30805984in}m{0.30805984in}m{0.30805984in}m{0.30525985in}}
\centering a &
\centering a &
\centering a &
\centering a &
\centering a &
\centering a &
\centering a &
\centering a &
\centering a &
\centering a &
\centering a &
\centering a &
\centering a &
\centering a &
\centering\arraybslash a\\
\centering a &
\centering a &
\centering a &
\centering b &
\centering b &
\centering a &
\centering b &
\centering b &
\centering a &
\centering b &
\centering b &
\centering b &
\centering b &
 b &
\centering\arraybslash b\\
\centering a &
\centering a &
\centering b &
\centering a &
\centering b &
\centering b &
\centering b &
\centering a &
\centering b &
\centering a &
\centering c &
\centering b &
\centering c &
\centering c &
\centering\arraybslash c\\
\centering a &
\centering b &
\centering a &
\centering a &
\centering b &
\centering b &
\centering a &
\centering b &
\centering c &
\centering c &
\centering a &
\centering c &
\centering b &
\centering c &
\centering\arraybslash d\\
\end{supertabular}
\end{flushleft}
One of these, abab, is special in that it is the only “non-planar” scheme. In this pattern, you cannot connect the a’s
with an arc and the b’s with an arc without the arcs crossing.

The planar rhyme schemes are enumerated by the Catalan numbers. These numbers are a subsequence of the Bell numbers,
which enumerate all rhyme schemes.

Thus far, we have the following bijections:

\section[Parentheses $\leftrightarrow $ H{}-D sequences]{Parentheses $\leftrightarrow $ H-D sequences}
\section[\ \ \ \ \ \ \ \ \ \ \ \ \ \ \ \ \ \ \ \ \ \ \ \ \ \ \ \ \ \ \ \ \ \ \ \ \ \ \ \ \ \ \ \ \ \ \ \ \ \ \ \ \ \ \ \ \ \ \ \ H{}-D
$\leftrightarrow $ Paths]{ H-D $\leftrightarrow $ Paths}
\section[\ \ \ \ \ \ \ \ \ \ \ \ \ \ \ \ \ \ \ \ \ \ \ \ \ \ \ \ \ \ \ \ \ \ \ \ \ \ \ \ \ \ \ Triangulations
$\leftrightarrow $ Parentheses]{ Triangulations $\leftrightarrow $ Parentheses}
 Insertions $\leftrightarrow $ Paths

\section[\ \ \ \ \ \ \ \ \ \ \ \ \ \ \ \ \ \ \ \ \ \ \ \ \ \ \ \ \ \ \ \ \ \ \ \ \ \ \ \ \ \ \ \ \ \ \ \ \ \ \ \ \ \ \ \ \ \ Trees
$\leftrightarrow $ Parentheses]{ Trees $\leftrightarrow $ Parentheses}
See if you can connect the rhyme schemes to any of the above.

\section[THE CATALAN NUMBERS]{THE CATALAN NUMBERS}
\begin{center}
\tablefirsthead{}
\tablehead{}
\tabletail{}
\tablelasttail{}
\begin{supertabular}{m{0.28095984in}m{0.28165984in}m{0.28095984in}m{0.28165984in}m{0.28095984in}m{0.28165984in}m{0.28095984in}m{0.28165984in}m{0.28095984in}m{0.28165984in}|m{0.28095984in}|m{0.28165984in}m{0.28095984in}m{0.28165984in}m{0.28095984in}m{0.28165984in}m{0.28095984in}m{0.28165984in}m{0.28095984in}m{0.28165984in}m{0.28095984in}m{0.27125984in}}
\hhline{~~~~~~~~~~-~~~~~~~~~~~}
 &
 &
 &
 &
 &
 &
 &
 &
 &
 &
\centering 1 &
 &
 &
 &
 &
 &
 &
 &
 &
 &
 &
\centering\arraybslash 1\\\hhline{~~~~~~~~~~-~~~~~~~~~~~}
 &
 &
 &
 &
 &
 &
 &
 &
 &
\multicolumn{1}{m{0.28165984in}}{\centering 1} &
\multicolumn{1}{m{0.28095984in}}{} &
\centering 1 &
 &
 &
 &
 &
 &
 &
 &
 &
 &
\\\hhline{~~~~~~~~~~-~-~~~~~~~~~}
 &
 &
 &
 &
 &
 &
 &
 &
\centering 1 &
 &
\centering 2 &
\multicolumn{1}{m{0.28165984in}|}{} &
\multicolumn{1}{m{0.28095984in}|}{\centering 1} &
 &
 &
 &
 &
 &
 &
 &
 &
\centering\arraybslash 1\\\hhline{~~~~~~~~~~-~-~~~~~~~~~}
 &
 &
 &
 &
 &
 &
 &
\centering 1 &
 &
\multicolumn{1}{m{0.28165984in}}{\centering 3} &
\multicolumn{1}{m{0.28095984in}}{} &
\centering 3 &
 &
\centering 1 &
 &
 &
 &
 &
 &
 &
 &
\\\hhline{~~~~~~~~~~-~-~~~~~~~~~}
 &
 &
 &
 &
 &
 &
\centering 1 &
 &
\centering 4 &
 &
\centering 6 &
\multicolumn{1}{m{0.28165984in}|}{} &
\multicolumn{1}{m{0.28095984in}|}{\centering 4} &
 &
\centering 1 &
 &
 &
 &
 &
 &
 &
\centering\arraybslash 2\\\hhline{~~~~~~~~~~-~-~~~~~~~~~}
 &
 &
 &
 &
 &
\centering 1 &
 &
\centering 5 &
 &
\multicolumn{1}{m{0.28165984in}}{\centering 10} &
\multicolumn{1}{m{0.28095984in}}{} &
\centering 10 &
 &
\centering 5 &
 &
\centering 1 &
 &
 &
 &
 &
 &
\\\hhline{~~~~~~~~~~-~-~~~~~~~~~}
 &
 &
 &
 &
\centering 1 &
 &
\centering 6 &
 &
\centering 15 &
 &
\centering 20 &
\multicolumn{1}{m{0.28165984in}|}{} &
\multicolumn{1}{m{0.28095984in}|}{\centering 15} &
 &
\centering 6 &
 &
\centering 1 &
 &
 &
 &
 &
\centering\arraybslash 5\\\hhline{~~~~~~~~~~-~-~~~~~~~~~}
 &
 &
 &
\centering 1 &
 &
\centering 7 &
 &
\centering 21 &
 &
\multicolumn{1}{m{0.28165984in}}{\centering 35} &
\multicolumn{1}{m{0.28095984in}}{} &
\centering 35 &
 &
\centering 21 &
 &
\centering 7 &
 &
\centering 1 &
 &
 &
 &
\\\hhline{~~~~~~~~~~-~-~~~~~~~~~}
 &
 &
\centering 1 &
 &
\centering 8 &
 &
\centering 28 &
 &
\centering 56 &
 &
\centering 70 &
\multicolumn{1}{m{0.28165984in}|}{} &
\multicolumn{1}{m{0.28095984in}|}{\centering 56} &
 &
\centering 28 &
 &
\centering 8 &
 &
\centering 1 &
 &
 &
\centering\arraybslash 14\\\hhline{~~~~~~~~~~-~-~~~~~~~~~}
 &
\centering 1 &
 &
\centering 9 &
 &
\centering 36 &
 &
\centering 84 &
 &
\multicolumn{1}{m{0.28165984in}}{\centering 126} &
\multicolumn{1}{m{0.28095984in}}{} &
\centering 126 &
 &
\centering 84 &
 &
\centering 36 &
 &
\centering 9 &
 &
\centering 1 &
 &
\\\hhline{~~~~~~~~~~-~-~~~~~~~~~}
\centering 1 &
 &
\centering 10 &
 &
\centering 45 &
 &
\centering 120 &
 &
\centering 210 &
 &
\centering 252 &
\multicolumn{1}{m{0.28165984in}|}{} &
\multicolumn{1}{m{0.28095984in}|}{\centering 210} &
 &
\centering 120 &
 &
\centering 45 &
 &
\centering 10 &
 &
\centering 1 &
\centering\arraybslash 42\\\hhline{~~~~~~~~~~-~-~~~~~~~~~}
\end{supertabular}
\end{center}
\section{}
\section[FIGURE 1]{FIGURE 1}
\section[TRIVALENT, PLANTED TREES WITH n LEAVES]{TRIVALENT, PLANTED TREES WITH n LEAVES}
\section{}
[Warning: Draw object ignored][Warning: Draw object ignored][Warning: Draw object ignored]

[Warning: Draw object ignored]

n=1\ \ \ \   n=2\ \ \ \ \ \ \ \ \ \  n=3

[Warning: Draw object ignored][Warning: Draw object ignored][Warning: Draw object ignored][Warning: Draw object
ignored][Warning: Draw object ignored]

 n=4

[Warning: Draw object ignored][Warning: Draw object ignored]

\ \ \ \ 1100100\ \ \ \ \ \ \ \ \ \ 1010100

\ \ \ \ ( ( ab ( cd\ \ \ \ \ \ \ \ \ \ ( a ( b ( cd

\ \ \ \ HHDDHD\ \ \ \ \ \ \ \ \ \ HDHDHD

 FIGURE 2

\begin{flushleft}
\tablefirsthead{}
\tablehead{}
\tabletail{}
\tablelasttail{}
\begin{supertabular}{|m{2.13726in}|m{2.13796in}|m{2.13796in}|}
\hline
\centering [Warning: Draw object ignored] &
 &
{\centering ((ab(cd))\par}

\centering\arraybslash 1100100\\\hline
\centering [Warning: Draw object ignored] &
 &
{\centering (((ab)c)d)\par}

\centering\arraybslash 1110000\\\hline
\centering [Warning: Draw object ignored] &
 &
{\centering (a(b(cd)))\par}

\centering\arraybslash 1010100\\\hline
\centering [Warning: Draw object ignored] &
 &
{\centering (a(bc)d))\par}

\centering\arraybslash 1011000 \\\hline
\centering [Warning: Draw object ignored] &
 &
{\centering ((a(bc))d)\par}

\centering\arraybslash 1101000\\\hline
\end{supertabular}
\end{flushleft}
FIGURE 3

PROBLEMS ON CATALAN NUMBERS

\begin{enumerate}
\item Use the recursion for Pn and compute P6, P7.
\item Use the closed formula for Pn and compute P6, P7.
\item Use G = P1x + P2x2 + … and the recursion for Pn to determine a closed formula for Pn.
\item List the workable sequences for n = 4.
\item List the P6 ways of parenthesizing abcdef.
\item Display a bijection between the 14 sequences in Problem 4 and the 14 products in Problem 5.
\item Find the nonworkable sequence associated with each:
\end{enumerate}
\begin{enumerate}
\item H H H D D H\ \ \ \ \ \ (c) D D D H H H H H
\item D H H H H D\ \ \ \ \ \ (d) D D H H D H H H\ \ \ \ \ \ 
\end{enumerate}
\begin{enumerate}
\item Draw the T6 triangulations of a convex hexagon.
\item Display a bijection between the 5 ways of parenthesizing abcd and the 5 triangulations of a convex pentagon.
\item Let  $C_n=\frac 1{n+1}\left(\genfrac{}{}{0pt}{0}{2n}n\right).$ Verifiy the formula:
\end{enumerate}
\begin{equation*}
C_n=\left(\genfrac{}{}{0pt}{0}n1\right)C_{n-1}-\left(\genfrac{}{}{0pt}{0}{n-1}2\right)C_{n-2}+\left(\genfrac{}{}{0pt}{0}{n-2}3\right)C_{n-3}-{\dots}
\end{equation*}
\begin{enumerate}
\item Verify that H3 = 5 by drawing the required paths.
\item Assign the appropriate binary sequences to each of the 5 Trees for n=3 in FIGURE 2.
\end{enumerate}
FOR EACH OF THE FOLLOWING, INVESTIGATE BIJECTIONS THAT RELATE ONE TO ANOTHER.

\begin{enumerate}
\item A and B each receive n votes. Let Vn denote the number of ways that the 2n votes can be tallied so that A never
trails B. Let V0 = 1. 
\item Place 2n points on the circumference of a circle and draw n nonintersecting chords in Dn ways; D0 = 1.
\item In how many ways can 1, 2, 3, …, 2n be inserted into a 2×n rectangle such that the entries are increasing in rows
and columns. Let the answer be Yn; Y1 = 1, Y2 = 2. 
\item Place  $2n$ points on a line segment and join them in pairs by nonintersecting arcs above the segment. In how many
ways can this be done? Call the answer Sn; S1=1, S2=2.
\item A rook on an  $n+1$ by  $n+1$ chessboard must move from the lower left corner to the upper right corner never
going above the diagonal. How many paths are possible? Let  $K_n$ be the answer. 
\item On an  $n+1$ by  $n+1$ chessboard a king  starts on the first row and moves one square forward or back along a
fixed column and ends on the starting square after  $2n$ moves. In how many ways,  $G_n$, can this be done? 
\item Count the number of planar rhyme schemes for a stanza consisting of n lines. Joanne Growney showed in her doctoral
thesis that the Bell numbers, which count all rhyme schemes, have the Catalan numbers as a subsequence and that these
enumerate precisely the planar rhyme schemes. For example, of the  $B_4=15$ rhyme schemes,  $C_4=14$ are planar. The
nonplanar one is given by abab. 
\end{enumerate}
WORKSHEET ON STIRLING NUMBERS OF THE 2ND KIND- $S(n,k)$

DEFINITION:  $S\left(n,k\right)$ is the number of ways of partitioning an n-set into k nonempty subsets. 

TASK 1: Compute  $S\left(3,1\right),S(4,1)$ and  $S(n,1)$. 

TASK 2: Compute  $S\left(2,2\right),S(3,2)$ and  $S(4,2)$.

TASK 3: Compute $S\left(3,3\right),S(4,3)$. 

TASK 4: Find formulas and give proofs for: 

\begin{equation*}
S\left(n,n\right)=?
\end{equation*}
\begin{equation*}
S\left(n,2\right)=?
\end{equation*}
\begin{equation*}
S\left(n,n-1\right)=?
\end{equation*}
\begin{equation*}
S\left(n,n-2\right)=?
\end{equation*}
TASK 5: Determine a recursion for  $S\left(n,k\right)$ by examining the following:

\ \ In computing  $S\left(4,3\right)$ look back at the two cases:

\begin{enumerate}
\item “Social”- The element 4 is part of another set. 
\end{enumerate}
\begin{enumerate}
\item “Antisocial”-The element 4 stands alone. 
\end{enumerate}
\begin{enumerate}
\item Deduce a recursion  $S\left(n,k\right)=?$
\end{enumerate}
TASK 6: Make the first 5 rows of the Stirling Triangle. 

STIRLING NUMBERS OF THE 2nd KIND

\ \ A distribution of 4 different (often we say distinguishable) objects  $1,2,3,4$ into 3 similar (often we say
indistinguishable) boxes, each of which becomes non-empty is modeled by partitioning  $\{1,2,3,4\}$ into 3 nonempty
subsets. The six ways of doing this are listed next:

\{4\} \{1\} \{2,3\}\ \ \ \ \{1,4\} \{2\} \{3\}

\{4\} \{2\} \{1,3\} \ \ \ \ \{1\} \{2,4\} \{3\}

\{4\} \{3\} \{1,2\} \ \ \ \ \{1\} \{2\} \{3,4\}

\ \ The Stirling numbers of the second kind, denoted by  $S(n,k)$, give the number of ways of partitioning an n-set into
k non-empty subsets. Here  $k{\in}\{1,2,{\dots}\}$ and $n{\geq}k$. The above example shows  $S\left(4,3\right)=6$. Also
notice that there are two cases: 4 is alone, or 4 is in some other batch; we can conclude that 
$S\left(4,3\right)=3S\left(3,3\right)+S(3,2)$ since if 4 is in a singleton set just partition the remaining 3 into 2
sets and otherwise 4 can join any of the 3 batches of 3 elements. 

\section[THEOREM 1: \ . ]{THEOREM 1:   $S\left(n,k\right)=\mathit{kS}\left(n-1,k\right)+S(n-1,k-1)$. }
Proof. We partition  $\{1,2,{\dots},n\}$ into k non-empty subsets in  $S(n,k)$ ways. If n is in a singleton set, there
are  $S(n-1,k-1)$ ways to partition the remaining   $n-1$ elements into  $k-1$ subsets. Otherwise, the element n can
join any of the k batches of   $n-1$ elements in  $\mathit{kS}(n-1,k)$ ways. 

\ \ Polya, in Notes on Introductory Combinatorics, refers to these as the “antisocial” and “social” cases. 

\ \ The reader should check that  $S\left(3,3\right)=1$ and that  $S\left(3,2\right)=3$ so that 
$S\left(4,3\right)=3{\bullet}1+3=6$.

Border Formulas: The reader should verify that  $S\left(n,1\right)=1$ and  $S\left(n,n\right)=1$ for  $n{\in}Z^{+?}$.
From 

these border formulas and the recursion in Theorem 1 the following Stirling’s Second Triangle can be made. 

\section[1]{1}
\ \ 1\ \ 1

\ \ 1\ \ 3\ \ 1

\ \ 1\ \ 7\ \ 6\ \ 1

\ \ 1\ \ 15\ \ 25\ \ 10\ \ 1

THEOREM 2.

\begin{enumerate}
\item \begin{equation*}
S\left(n,n-1\right)=\left(\begin{matrix}n\\2\end{matrix}\right)
\end{equation*}
\item \begin{equation*}
S\left(n,2\right)=2^{n-1}-1
\end{equation*}
\item \begin{equation*}
S\left(n,n-2\right)=\left(\begin{matrix}n\\3\end{matrix}\right)+3\left(\begin{matrix}n\\4\end{matrix}\right)
\end{equation*}
\item \begin{equation*}
S\left(n,n-3\right)=\left(\begin{matrix}n\\4\end{matrix}\right)+10\left(\begin{matrix}n\\5\end{matrix}\right)+15\left(\begin{matrix}n\\6\end{matrix}\right)
\end{equation*}
\end{enumerate}
Proof. You are asked to prove these in the problems. 

\ \ The entries in the Pascal Triangle,  $\left(\begin{matrix}n\\k\end{matrix}\right)$, satisfy the nice recursion 
$\left(\begin{matrix}n\\k\end{matrix}\right)=\left(\begin{matrix}n-1\\k\end{matrix}\right)+\left(\begin{matrix}n-1\\k-1\end{matrix}\right)$
 and also have a closed formula:  $\left(\begin{matrix}n\\k\end{matrix}\right)=n!/k!\left(n-k\right)!$. Theorem 1 gives
us a recursion for the Stirling numbers  $S\left(n,k\right)$ and, unfortunately, the best we can do for a “closed”
formula is contained in the following. 

\section[THEOREM 3.]{THEOREM 3.  $S\left(n,k\right)=\frac
1{k!}\left[k^n-\left(\begin{matrix}k\\1\end{matrix}\right)\left(k-1\right)^n+\left(\begin{matrix}k\\2\end{matrix}\right)\left(k-2\right)^n-\left(\begin{matrix}k\\3\end{matrix}\right)\left(k-3\right)^n+{\dots}+\left(-1\right)^{k-1}\left(\begin{matrix}k\\k-1\end{matrix}\right)1^n\right]$
}
Proof (by Polya). Suppose you wished to paint n houses and you have k different colors available. The first house can be
painted  in k different ways, the second  in k different ways, etc. So there are  $k^n$ ways. How many ways actually
use all k colors? Let  $\alpha _i$ be the property that no house is painted with the  $i^{th}$ color, and let 
$N(\alpha _i)$ denote the number of ways of painting the n houses without using the  $i^{th}$ color. Analogously for 
$N\left(\alpha _i,\alpha _j\right),N\left(\alpha _i,\alpha _j,\alpha _k\right),$ etc. Since   $N\left(\alpha
_i\right)=\left(k-1\right)^n$,  $N\left(a_i,\alpha _j\right)=\left(k-2\right)^n,{\dots}$ the number of ways of painting
n houses using all k colors is, using PIE, 
$k^n-\left(\begin{matrix}k\\1\end{matrix}\right)\left(k-1\right)^n+\left(\begin{matrix}k\\2\end{matrix}\right)\left(k-2\right)^n-\left(\begin{matrix}k\\3\end{matrix}\right)\left(k-3\right)^n+{\dots}+\left(-1\right)^k\left(\begin{matrix}k\\k\end{matrix}\right)0^n$.
Alternately, we could first partition the n houses into k (non-empty) sets, and then paint each set. This can be
accomplished in  $k!S\left(n,k\right)$ ways. Now equate these two expressions. 

\section[COROLLARY. \ \ \ \ \ ]{COROLLARY.   $S\left(n,3\right)=\frac 1 2\left[3^{n-1}-2^n+1\right]$ }
THEOREM 4.  
$x^n=S\left(n,1\right)x+S\left(n,2\right)x\left(x-1\right)+S\left(n,3\right)x\left(x-1\right)\left(x-2\right)+{\dots}$

\begin{equation*}
{\dots}+S\left(n,n\right)x\left(x-1\right){\dots}(x-n+1)
\end{equation*}
Proof (by Polya). We can paint n houses, with x colors available, in  $x^n$ ways. Using exactly one color, there are 
$S\left(n,1\right)x$ ways. Using exactly two colors there are  $S\left(n,2\right)x(x-1)$ ways. Continue in this manner.


It is in this form that James Stirling originally developed these numbers.  $S(n,k)$ is used to convert from powers to
binomial coefficients as shown in the following: 

\section[THEOREM 5. \ \ \ \ \ \ \  ]{THEOREM 5.  
$x^n=S\left(n,1\right)\left(\begin{matrix}x\\1\end{matrix}\right)1!+S\left(n,2\right)\left(\begin{matrix}x\\2\end{matrix}\right)2!+{\dots}+S\left(n,n\right)\left(\begin{matrix}n\\n\end{matrix}\right)n!$
}
\section{}
\section{}
\section{}
\section{}
\section{}
\section{}
\section{}
\section[]{}
\section{}
\section{}
\section{}
\section{}
\section{}
\section{}
\section{}
\section[PROBLEMS]{PROBLEMS}
\begin{enumerate}
\item Compute by listing the subsets  $S(4,2)$ and  $S(5,2)$.
\item Make the 6th row of the Stirling Triangle. 
\item Prove the four parts of THEOREM 2. 
\item List the sequence of elements that are row sums of the Stirling Triangle. 
\item How many subsets \{a, b, c\} are there of \{2, 3, 4, …\} such that 
$\mathit{abc}=2{\bullet}3{\bullet}5{\bullet}7{\bullet}11{\bullet}13{\bullet}17$?
\item Let  $S=\{2,3,4,{\dots}\}$. How many ordered triples (a, b, c) are there in S x S x S such that
\end{enumerate}
  $\mathit{abc}=2{\bullet}3{\bullet}5{\bullet}7{\bullet}11{\bullet}13{\bullet}17$?

\begin{enumerate}
\item Compute each using THEOREM 3. 

\begin{enumerate}
\item \begin{equation*}
S\left(5,3\right)
\end{equation*}
\item \begin{equation*}
S(6,3)
\end{equation*}
\item \begin{equation*}
S(n,2)
\end{equation*}
\item \begin{equation*}
S(n,3)
\end{equation*}
\end{enumerate}
\item Write out the proof of THEOREM 3 for k=3.
\item Express x3 as suggested in THEOREM 4. Also, multiply your answer out to verify. 
\item Use your result in Problem 9 to express x3 as a linear combination of the binomial coefficients 
$\left(\begin{matrix}x\\1\end{matrix}\right),\left(\begin{matrix}x\\2\end{matrix}\right),\left(\begin{matrix}x\\3\end{matrix}\right)$.

\item Determine the following:

\begin{enumerate}
\item  $a,b,c$ so that 
$n^4=24\left(\begin{matrix}n\\4\end{matrix}\right)+6a\left(\begin{matrix}n\\3\end{matrix}\right)+2b\left(\begin{matrix}n\\2\end{matrix}\right)+c\left(\begin{matrix}n\\1\end{matrix}\right)$.
\end{enumerate}
\end{enumerate}
\begin{enumerate}
\item \begin{enumerate}
\item  $a,b,c,d$ so that 
$n^5=5!\left(\begin{matrix}n\\5\end{matrix}\right)+a\left(\begin{matrix}n\\4\end{matrix}\right)+b\left(\begin{matrix}n\\3\end{matrix}\right)+c\left(\begin{matrix}n\\2\end{matrix}\right)+d\left(\begin{matrix}n\\1\end{matrix}\right)$.
 
\end{enumerate}
\item Express  $1^4+2^4+3^4+{\dots}+n^4$ as a polynomial in n. 
\item The numbers  $S(n,k)$ and  $k!S(n,k)$ are solutions to two different distribution problems. Describe each. 
\item Prove THEOREM 5. 
\item Why is  $4^n-4{\bullet}3^n+6{\bullet}2^n-4$ always divisible by 24?
\end{enumerate}
BELL NUMBERS

\ \ The Bell number,  $B_n$, denotes the number of ways that a set of n objects S can be partitioned into nonempty
subsets. The definition of “partition” implies two properties: The subsets are disjoint and their union is S. 

\ \ Essential properties and facts about  $B_n$ are summarized next. 

\section{}
\section[BELL NUMBERS]{BELL NUMBERS}
\section[1, 1, 2, 5, 15, 52, 203, 877, 4040, 21147, 115975, …]{1, 1, 2, 5, 15, 52, 203, 877, 4040, 21147, 115975, …}
\section[FORMULA: \ \ ]{FORMULA: \ \ }
\section[\ ]{  $B_n=S\left(n,1\right)+S\left(n,2\right)+S\left(n,3\right)+{\dots}+S(n,n)$}
\section[ ]{ $B_n=\frac 1 e\sum _{k=0}^{{\infty}}\frac{k^n}{k!}(\mathit{Dobinsk}i^'s\mathit{Formula})$ }
\section[ ]{ $B_n=L\left(x^n\right)(\mathit{Gian}-\mathit{Carlo}\mathit{Rota})$ }
\section{}
RECURSION: 
$B_n=\left(\begin{matrix}n-1\\0\end{matrix}\right)B_0+\left(\begin{matrix}n-1\\1\end{matrix}\right)B_1+{\dots}+\left(\begin{matrix}n-1\\n-1\end{matrix}\right)B_{n-1},B_0=1$

E.G.F.:   $e^{e^x-1}=\sum _{k=0}^{{\infty}}B_n\frac{x^n}{n!}$

\section[BELL TRIANGLE: \ \ 1]{BELL TRIANGLE: \ \ 1}
\ \ \ \ \ \ 1\ \ 2

\ \ \ \ \ \ 2\ \ 3\ \ 5

\ \ \ \ \ \ 5\ \ 7\ \ 10\ \ 15

\ \ \ \ \ \ 15\ \ 20\ \ 27\ \ 37\ \ 52

\ \ \ \ \ \ …

Two observations: The first few Bell’s look like the Catalans; the sequence of Bells increases in size rapidly. The Bell
numbers can be generated by constructing what is called the Bell Triangle. To construct this triangle, begin with a 1
at the top and a 1 below it. Add these two numbers together and put the sum 2, to the right of the 1 in the second
column. This 2 is also the first entry of the third row. The second entry in the third row is found by adding the 2 to
the number 1 above it. This sum is 3 and goes to the right of the 2. The 3 is now below a 2. Adding these two numbers
produces the last number, 5, in the third row. Since 5 has no number above it, the third row is complete. 5 now becomes
the first number in the fourth row and the process continues. 

Construction of the triangle follows two basic rules:

\begin{enumerate}
\item The last number of each row is the first number of the next row. 
\item All other numbers are found by adding the number to the left of the missing number to the number directly above
this same number.
\end{enumerate}
When the triangle is extended, as above, the Bell Numbers are found down the first column as well as along the outside
diagonal.

\ \ \ \ \ \ 1

\ \ \ \ \ \ 1\ \ 2

\ \ \ \ \ \ 2\ \ 3\ \ 5

\ \ \ \ \ \ 5\ \ 7\ \ 10\ \ 15

\ \ \ \ \ \ 15\ \ 20\ \ 27\ \ 37\ \ 52

\ \ \ \ \ \ 52\ \ 67\ \ 87\ \ 114\ \ 151\ \ 203

\ \ \ \ \ \ 203\ \ 255\ \ 322\ \ 409\ \ 523\ \ 674\ \ 877

As with Pascal’s Triangle, the Bell triangle has several interesting properties. If the sum of a row is added to the
Bell number at the end of that row, the next Bell number is obtained. For example, the sum of the fourth row plus the
Bell number at the end of the row: 

15 + 20 + 27 + 37 + 52 + 52 is 203, the next Bell number. Also, the numbers of the second diagonal 1, 3, 10, 37, 151,
674, … are the sums of the horizontal rows. Rotating the triangle slightly creates a difference triangle analogous to
Pascal’s Triangle. The entries that are formed recursively by adding in Pascal’s Triangle are now differences of the
two numbers above them in the Bell triangle.

\ \ \ \ 1\ \ 2\ \ 5\ \ 15\ \ 52\ \ 203\ \ 877…

\ \ \ \   1\ \   3\ \   10  37  151  674…

\ \ \ \ \ \ 2\ \ 7\ \ 27\ \ 114\ \ 523…

\ \ \ \ \ \   5  20  87  409…

\ \ \ \ \ \ \ \ 15\ \ 67\ \ 322…

\ \ \ \ \ \ \ \   52\ \   255…

\ \ \ \ \ \ \ \ \ \ 203…

Finally, rewriting the Bell triangle recursively indicates a nice connection of the Bell Numbers to Pascal’s triangle. 

\begin{equation*}
B_0=B_1
\end{equation*}
 $B_1$   $B_0+B_1=B_2$

 $B_2$   $B_1+B_2$   $B_0+2B_1+B_2=B_3$

 $B_3$   $B_2+B_3$   $B_1+B_2+B_2+B_3$\ \  $B_0+3B_1+3B_2+B_3=B_4$

This pattern suggests the  $(n+1)^{\mathit{th}}$ Bell Number can be represented recursively by 


$B_{n+1}=\left(\begin{matrix}n\\0\end{matrix}\right)B_0+\left(\begin{matrix}n\\1\end{matrix}\right)B_1+\left(\begin{matrix}n\\2\end{matrix}\right)B_2+{\dots}+\left(\begin{matrix}n\\n\end{matrix}\right)B_n$.
The coefficients of this equation are the entries of the nth row of Pascal’s triangle. You are asked to prove this
recursion in the exercises.

\ \ The Bell numbers are related to the Stirling numbers and can be defined as the sum of the Stirling numbers of the
second kind. That is,  $B_n=S\left(n,1\right)+S\left(n,2\right)+{\dots}+S(n,n)$, where

 S(n, k) represents the number of ways of grouping n elements into k subsets. So 

 $B_3=S\left(3,1\right)+S\left(3,2\right)+S\left(3,3\right)=1+3+1=5.$ Since the Bell numbers count the partitions of a
set of elements, they are used in prime-number theory to enumerate the number of ways to factor a number with distinct
prime factors. For example, 42 has three distinct prime factors: 2, 3, and 7. Since  $B_3=5$, we know there are five
ways of factoring 42. These are 

2 x 3 x 7, 2 x 21, 3 x 14, 6 x 7, and 42. So the number 210, which has 4 distinct factors, 2, 3, 5, and 7, can be
factored in  $B_4=15$ ways. 

\ \ The Bell numbers can be used to model many real-life situations. For example, the number of different ways two
people can sleep in unlabeled twin beds is  $B_2=2$ ways: They can sleep in the same bed or in separate beds. The
number of ways of serving a dinner consisting of three items, such as a salad, bread, and fish is  $B_3=5$ ways: each
could be served on a separate plate, salad and bread could be on one plate and fish on another, salad and fish on one
plate and bread on another, or all three items could be on the same plate. This example serves as a model for the five
ways three people can occupy three unlabeled beds, the five ways three prisoners can be handcuffed together, the five
ways three nations can be form alliances, or any situation of partitioning three distinct elements into non-empty
subsets. 

\ \ One interesting application of Bell numbers is in counting the number of rhyme schemes possible for a stanza in
poetry. There are  $B_2=2$ possibilities for a two-line stanza: the lines can either rhyme or not rhyme. The possible
rhyme schemes of a three-line stanza can be described as aaa, aab, aba, abb, and abc: thus there are 5 or  $B_3$
possible rhyme schemes. The Japanese used diagrams depicting possible rhyme schemes of a five-line stanza  $B_5=52$ as
early as 1000 A.D. in the Tale of the Genji by Lady Shikibu Murasaki. 

PROBLEMS FOR BELL NUMBERS

The Bell number (named after Eric Temple Bell, a Scottish-born American mathematician, who died in 1960)  $B_n$ is the
number of ways of partitioning an n-set into subsets.  For example,  $B_3=5$; the 5 ways of partitioning the elements
of  \{1, 2, 3\} are

\{1\} \{2\} \{3\}\ \ \{1, 2\} \{3\}\ \ \{1, 3\} \{2\}\ \ \{2, 3\} \{1\}\ \ \{1, 2, 3\}

\begin{enumerate}
\item Compute  $B_1,B_2,$ and  $B_4$ by listing all the relevant partitions.
\end{enumerate}
\begin{enumerate}
\item Explain why  $B_n=S\left(n,1\right)+S\left(n,2\right)+{\dots}+S(n,n)$.
\end{enumerate}
\begin{enumerate}
\item Compute  $B_5,B_6,B_7,B_8$ using a table of Stirling numbers  $S(n,k)$ and the formula in Problem 2.
\end{enumerate}
\begin{enumerate}
\item Prove that 
$B_n=\left(\begin{matrix}n-1\\0\end{matrix}\right)B_0+\left(\begin{matrix}n-1\\1\end{matrix}\right)B_1+\left(\begin{matrix}n-1\\2\end{matrix}\right)B_2+{\dots}+\left(\begin{matrix}n-1\\n-1\end{matrix}\right)B_{n-1}$.
\end{enumerate}
[Here we let  $B_0=1$]. HINT: First try n=5 and partition  $\{1,2,3,4,5\}$ by treating 5 as “special.” The element 5 can
be a singleton, in a doubleton, or … .

\begin{enumerate}
\item Prove that
\end{enumerate}
\begin{equation*}
S_n\left(n+1,r\right)=\left(\begin{matrix}n\\0\end{matrix}\right)S\left(0,r-1\right)+\left(\begin{matrix}n\\1\end{matrix}\right)S\left(1,r-1\right)+\left(\begin{matrix}n\\2\end{matrix}\right)S\left(2,r-1\right)+{\dots}
\end{equation*}
\begin{equation*}
+\left(\begin{matrix}n\\n\end{matrix}\right)S(n,r-1)
\end{equation*}
\begin{enumerate}
\item Using the .results in Problems 2 and 5 prove the recursion in Problem 4. 
\end{enumerate}
\begin{enumerate}
\item The set  $\left\{u_0,u_1,u_2,{\dots}\right\}$ where 
$u_0=1,u_1=x,u_2=x\left(x-1\right),u_3=x\left(x-1\right)\left(x-2\right),{\dots}$ is a basis for the vector space V of
all polynomials with real coefficients. Let
\end{enumerate}
 $P\left(x\right)=c_0u_0+c_1u_1+c_2u_2+{\dots}$ be any element of V, and define the functional L as follows: 
$L\left[P\left(x\right)\right]=c_0+c_1+{\dots}$ .

\begin{enumerate}
\item \begin{enumerate}
\item Prove that 
$L\left[P\left(x\right)+Q\left(x\right)\right]=L\left[P\left(x\right)\right]+L\left[Q\left(x\right)\right].$
\item Prove that  $L\left[\mathit{rP}\left(x\right)\right]=\mathit{rL}[P\left(x\right)]$ for  $r{\in}$ Reals. 
\item Prove that  $B_n=L[x^n]$. 
\item Prove that  $L\left[x^{n+1}\right]=L[\left(x+1\right)^n]$.
\item Derive the recursion in Problem 4 using (d). 
\end{enumerate}
\end{enumerate}
\section[Tales of Statisticians]{Tales of Statisticians}
ERIC TEMPLE BELL

7 Feb 1883-20 Dec 1960

Bell was no statistician; rather, a number theorist. He was and remains one of the great mathematical statesmen. His
book Men of Mathematics has inspired scores to take up a mathematical career, and revealed to uncounted thousands what
a mathematical life is all about in the first place. 

His own early life was a mystery until parts of it were unraveled by Constance Reid. He was born in Aberdeen in 1883,
spent his early years in London, and came to America at the age of nineteen in 1903. He finished college at Stanford in
1904, and went on to the University of Washington (MS 1907) and Columbia (PhD 1912). He taught at the University of
Washington from that year until1926. His contribution to statistics in this period was to encourage Harold Hotelling to
switch from journalism to math (Hotelling named his son after Bell). In his own researches at this time, Bell produced
significant results in number theory, including Diophantine analysis: equations whose solutions are limited to whole
numbers. The Bell numbers (1, 1, 2, 5, 15, 52, and so on), a series of whole numbers arising in the theory of
partitions, were first investigated by him, and are named for him. His 1921 memoir Arithmetical Paraphrases won him the
prestigious Bocher Prize (jointly with Solomon Lefschetz) in 1924. 

In that year he began a second life as a science fiction writer, under the pen name John Taine (his only son, born in
1917, had been named Taine Bell). In all, he produced twelve novels and several stories between 1924 and 1954. In 1926,
having earlier declined offers from Chicago and Columbia, he joined the faculty of the California Institute of
Technology, where he became a distinctive and even eccentric figure, and had much to do with building up the quality of
the mathematics faculty. His next mathematical publication was the comprehensive survey Algebraic Arithmetic (1927).
His enduring masterpiece of popularization, Men of Mathematics, came somewhat later, in 1937. The more technically
advanced Development of Mathematics (1940) had a comparable appeal and influence for professional and preprofessional
readers. His gift of explanation and his fascination with the integers are both evident in the Magic of Numbers (1946),
a book on Pythagoras and the number mysticism of the Pythagorean school. 

Bell retired from CalTech in 1951, and died at the age of 77 in 1960. With the assistance of D.H. Lehmer, who had
provided the otherwise missing final chapter, he had just completed a book entitled The Last Problem. It was about
Fermat, to whom, and to whose tantalizing unproven Last Theorem, a chapter of Men of Mathematics had already been
devoted. Bell’s conviction at the time, that Fermat’s unmatched insight into the character of numbers did not mislead
him as to the existence of a proof, was vindicated recently when a proof was finally found. 

Bell’s infectious love of number, his sense of the large sweep of mathematical history, and his generous anger at the
fools and poopheads who all too densely populate that history, are his great legacy to posterity. 

\section[Postscript 2003]{Postscript 2003}
The above notice happened to catch the eye of ETB’s grandson, Lyle Bell, the only son of Taine Bell, who commented that
ETB “would have loved the last paragraph.” We asked Lyle what ETB was like as a grandfather. Here, with his permission,
is his answer, along with notes on ETB’s last days. WE should add by way of clarification that “Romps” was ETB’s name
for himself, and “Toby” was his name for his wife Jessie. 

“E.T. Bell, known as Pop Romps to me (he refused to be called Grandpa), brings back some interesting and humorous
memories. You asked how he was as a grandfather. The answer is that he tolerated me and my two sisters reasonably well.
He wasn’t the type to sit his grandkids on his lap. I believe he was pretty fond of Laurie, the older of the two (both
younger than me). From what I know, he always took more of a liking to females than males (that reminds me of the fact
that he apparently did not want any children, his wife “tricked” him into it, and he was unhappy that he had a boy
rather than a girl). We visited him in Pasadena maybe two or three times that I can remember. I remember a modest home
with walls full of books and lots of cigar smoke. He had quite a large garden and a really neat study in a small
structure in the middle of the garden. He visited us in Watsonville a few times when he was healthy. I remember one
time when I felt I really hadn’t spent any time with him, and asked to stay home from school for the day. We sat in the
living room together, no doubt awkwardly for him. I don’t remember anything about the visit except that I was just
learning to multiply at the time, and he asked me to figure out what 7 x 7 is. It took me a while to get 49, and to
this day those are my favorite numbers. 

Some time in late 1958  or early 1959 he was still living in Pasadena when he fell off a chair he was standing on to
change a light bulb and broke his arm. My parents (both physicians) decided to bring him up to the Watsonville Hospital
where they could keep an eye on him while he recovered. Then they decided he wasn’t fit to live by himself and that he
should live in a rest home in Watsonville. When they announced their plans, he made is abundantly clear that he had no
intention of living with a bunch of old farts. He would stay right where he was, end of discussion! He was moved to a
private room close to the nurses’ station. He flirted with the nurses, smoked cigars, filled his room with books,
worked on The Last Problem, and drank Scotch. He was quite popular with the nurses. On one occasion he caught his bed
on fire as a result of careless smoking. He was reprimanded by the hospital administration and told he could not smoke
without a nurse being present. He said, “Great, I will enjoy more company!” and continued to smoke when he wanted to. 

I was in the sixth grade when he passed away. As I was getting on my bicycle to ride to school, my Mom came out to tell
me that Pop Romps had died the night before. She told me to go to school. That evening I remember having a short,
awkward conversation with my Dad and telling him I was sorry that Pop Romps had died. There was NO other discussion or
acknowledgement of his death within the family at the time. There was no way that my Dad would get involved in with
kind of service because his religious upbringing was totally non-existent. When he was in the third grade, he asked his
parents what the big plus sign on the building was for! 

When I was in college, it occurred to me that I never knew what happened after Pop Romps died. I asked my Mom, and she
told me he was still in her closet. He had left very specific instructions that he was to be cremated and his ashes
spread at the base of the rock on a hillside in Yreka, California, where he had proposed to his wife, Toby. He had
spread her ashes there. He left directions and a photo of the location. Yreka is in the northernmost part of
California. My parents did not have an occasion to go up there and didn’t want to make a special trip, so he stayed in
the closet. They made the trip in the early 1970’s. They arrived early on a foggy morning and found the rock near a
swing set in the back yard of a nice home. Dad hopped the fence and did the deed. The ashes were very light in color
and contained bone fragments. They didn’t exactly blend with the green grass, which was covered with dew. He tried to
work them into the grass with his feet and just made matters worse. He got back in the car and drove out of town fast
before somebody spotted the rather suspicious activity. Pop Romps was no doubt reveling through the whole ordeal.”

These pages are copyright 2001-by E. Bruce Brooks

\section[Group Projects{}- Math 528]{Group Projects- Math 528}
\section{}
\begin{enumerate}
\item Paths and Binomial Coefficients
\item On Triangular Numbers
\item Fibonacci Numbers
\item Generalized Pascal Triangles
\item A Pascal-like Triangle of Eulerian Numbers
\item Leibnitz Harmonic Triangle
\item The Twelve Days of Christmas
\end{enumerate}
\section[PROJECT 1: PATHS AND BINOMIAL COEFFICIENTS]{PROJECT 1: PATHS AND BINOMIAL COEFFICIENTS}
\begin{enumerate}
\item A path consists of vertical and horizontal line segments of length 1. Determine the number of paths

\begin{enumerate}
\item of length 7 from the origin $\left(0,0\right)$ and  $(4,3)$.
\item of length  $a+b$ from  $\left(0,0\right)$ and  $(a,b)$.
\end{enumerate}
\item A lattice point in the plane is a point  $(m,n)$ with  $m,n{\in}Z$.

\begin{enumerate}
\item How many paths of length 4 are there from  $(0,0)$ to lattice points on the line  $y=-x+4$? 
\item Determine the number of paths of length 10 from  $(0,0)$ to the lattice points on  $y=10-x$.
\end{enumerate}
\item Give a geometrical (path argument) for the identity 
$2^n=\left(\begin{matrix}n\\0\end{matrix}\right)+\left(\begin{matrix}n\\1\end{matrix}\right)+{\dots}+\left(\begin{matrix}n\\n\end{matrix}\right)$
using the idea in \#2. 
\item Generalize \#1 to three dimensions; introduce obstructions; use inclusion-exclusion. 
\item Give a path proof for 
$\left(\begin{matrix}n\\k\end{matrix}\right)=\left(\begin{matrix}n-1\\k\end{matrix}\right)+\left(\begin{matrix}n-1\\k-1\end{matrix}\right)$.

\item How many lattice paths from  $\left(0,0\right)$ to all lattice points on the line  $x+2y=n$ are there?
\item Determine the number of lattice paths from  $\left(0,0,0\right)$ to lattice points on that portion of the plane 
$x+y+z=2$ that lies in the first octant. 
\item Generalize \#7.
\item Determine the number of lattice paths from  $\left(0,0\right)$ to  $(n,n)$ that do not cross over (they may touch)
the line $y=x$.
\end{enumerate}
\section[PROJECT 2: ON TRIANGULAR NUMBERS]{PROJECT 2: ON TRIANGULAR NUMBERS}
\section[Let  \ denote the  triangular number.]{Let  $T_n=\frac{n\left(n+1\right)} 2$  denote the  $n^{\mathit{th}}$
triangular number.}
\begin{enumerate}
\item Give a geometric interpretation for   $T_n+T_{n+1}=\left(n+1\right)^2.$
\item Verify that   $T_n+T_m+\mathit{mn}=T_{m+n}$ algebraically. 
\item Give a geometric interpretation of the formula in \#2. 
\item Verify that  $T_n^2+T_{n-1}^2=T_{n^2}$.
\item Verify the following:

\begin{enumerate}
\item \begin{equation*}
3T_n+T_{n+1}=T_{2n+1}
\end{equation*}
\item  $3T_n+T_{n-1}=T_{2n}$\ \ (also give a geometrical proof)
\end{enumerate}
\item Determine the values  $T_6,T_{66},T_{666},T_{6666},{\dots}$.  Make a general statement and prove it. 
\item Determine the formula for  $T_3+T_6+T_9+{\dots}+T_{3n}.$
\item Verify:

\begin{enumerate}
\item \begin{equation*}
1+3+6+10+15=1^2+3^2+5^2
\end{equation*}
\item \begin{equation*}
1+3+6+10+15+21=2^2+4^2+6^2
\end{equation*}
\item Generalize and prove your generalization. 
\end{enumerate}
\end{enumerate}
[Warning: Draw object ignored]

\section[PROJECT 3: FIBONACCI NUMBERS]{PROJECT 3: FIBONACCI NUMBERS}
\begin{enumerate}
\item Briefly address history. 
\item Using a variety of proof methods (geometric, induction, matrices, recursion, telescoping sum) prove identities
such as:

\begin{enumerate}
\item \begin{equation*}
F_1+F_2+{\dots}+F_n=F_{n+2}-1
\end{equation*}
\item \begin{equation*}
F_1^2+F_2^2+{\dots}+F_n^2=F_nF_{n+1}
\end{equation*}
\item \begin{equation*}
F_{n+1}^2-F_nF_{n+2}=\left(-1\right)^n
\end{equation*}
\item \begin{equation*}
F_{n+2}=\left(\begin{matrix}n+1\\0\end{matrix}\right)+\left(\begin{matrix}n\\1\end{matrix}\right)+\left(\begin{matrix}n-1\\2\end{matrix}\right)+{\dots}
\end{equation*}
\end{enumerate}
\item Prove the “Binet Formula,”  $F_n=\frac{\alpha ^n-\beta ^n}{\alpha -\beta }$, where  $\alpha ,\beta $ satisfy 
$x^2-x-1=0$.  

\begin{enumerate}
\item By induction.
\item Using generating series.
\end{enumerate}
\item Use the Binet Formula to prove 
$\left(\begin{matrix}n\\0\end{matrix}\right)F_0+\left(\begin{matrix}n\\1\end{matrix}\right)F_1+{\dots}+\left(\begin{matrix}n\\n\end{matrix}\right)F_0=F_{2n}$
\item Investigate the geometric interpretations of the  $F_n$.
\item Prove that 
$F_n=\left[\left(\begin{matrix}n\\1\end{matrix}\right)+\left(\begin{matrix}n\\3\end{matrix}\right)5+\left(\begin{matrix}n\\5\end{matrix}\right)5^2{\dots}\right]+\left[\left(\begin{matrix}n\\0\end{matrix}\right)+\left(\begin{matrix}n\\2\end{matrix}\right)+\left(\begin{matrix}n\\4\end{matrix}\right)+{\dots}\right].$
\item Let  $Q=\left(\begin{matrix}1&1\\1&0\end{matrix}\right)$. Prove: 

\begin{enumerate}
\item  $Q^n=\left(\begin{matrix}F_{n+1}&F_n\\F_n&F_{n-1}\end{matrix}\right)$.
\item  $Q^2=Q+I$; use  $Q^{2n}$ to prove the identity in \#4. 
\item Use  $Q^{2n+1}=Q^nQ^{n+1}$ to show  $F_{2n+1}=F_{n+1}^2-F_n^2$ and other identities. 
\end{enumerate}
\item Show that if  $x^2=x+1$ then  $x^n=F_nx+F_{n-1}$ for $n{\geq}2$. Use this result to prove the Binet Formula,
namely that  $F_n=\frac{\alpha ^n-\beta ^n}{\alpha -\beta }$.
\item Investigate Lucas, Tribonacci, Tetranacci numbers. 
\end{enumerate}
References:  The Fibonacci Quarterly; The Golden Section by Garth E. Runion; The Divine Proportion by H.E. Huntley

PROJECT 4: GENERALIZED PASCAL TRIANGLES

\begin{enumerate}
\item Expand $\left(1+x+x^2\right)^2,\left(1+x+x^2\right)^3,\left(1+x+x^2\right)^4$.
\item Organize the coefficients as in the following triangle: 
\end{enumerate}
\ \ 1

\ \ 1\ \ 1\ \ 1

\ \ 1\ \ 2\ \ 3\ \ 2\ \ 1

1\ \ 3\ \ 6\ \ 7\ \ 6\ \ 3\ \ 1

\ \ 1\ \ 4\ \ 10\ \ 16\ \ 19\ \ 16\ \ 10\ \ 4\ \ 1

\begin{enumerate}
\item Investigate the properties of this triangle:

\begin{enumerate}
\item Produce the next row.
\item Give a recursion that produces elements in this triangle.
\item What are the row sums? Prove it!
\item What is the sum  $1^2+2^2+3^2+2^2+1^2$? Generalize and prove.
\item Is there a hockey stick theorem?
\end{enumerate}
\item Investigate 3-dimensional versions of  $\left(1+x+x^2\right)^n$ and  $\left(x+y+z\right)^n.$
\item Investigate  $\left(1+x+x^2+x^3\right)^n$.
\item Investigate  $\left(1+x+x^2+{\dots}+x^k\right)^n$.  
\item Where do the “Fibonacci” numbers get involved in the Pascal triangle?
\item Where do the “Tribonacci” numbers appear in the triangle in \#2?
\end{enumerate}
\section[PROJECT 5: A PASCAL{}-LIKE TRIANGLE OF EULERIAN NUMBERS]{PROJECT 5: A PASCAL-LIKE TRIANGLE OF EULERIAN NUMBERS}
\section{}
\begin{enumerate}
\item Show algebraically and geometrically that
$\left(\begin{matrix}n\\2\end{matrix}\right)+\left(\begin{matrix}n+1\\2\end{matrix}\right)=n^2$. 
\item Show that 
$\left(\begin{matrix}n\\3\end{matrix}\right)+4\left(\begin{matrix}n+1\\3\end{matrix}\right)+\left(\begin{matrix}n+2\\3\end{matrix}\right)=n^3$.
\item Use \#1 and the hockey stick theorem to find a nice formula for 
\end{enumerate}
 $1^2+2^2+3^2+{\dots}+n^2$. Contrast with the version usually seen in calculus. 

\begin{enumerate}
\item Use \#2 to find a nice formula for  $1^3+2^3+3^3+{\dots}+n^3$.
\item Express  $n^4$ as in problems \#1 and \#2. 
\item Use \#5 to find a nice formula for  $1^4+2^4+3^4+{\dots}+n^4$.
\item Here is an alternate way of accomplishing \#6:
\end{enumerate}
\ \ Find integers a, b, c, and d so that 
$1^4+2^4+3^4+{\dots}+n^4=a\left(\begin{matrix}n+1\\5\end{matrix}\right)+b\left(\begin{matrix}n+2\\5\end{matrix}\right)+c\left(\begin{matrix}n+3\\5\end{matrix}\right)+d\left(\begin{matrix}n+4\\5\end{matrix}\right)$

 using n=1, 2, 3, 4. This sometimes referred to as “Polynomial Fitting.”

\begin{enumerate}
\item Express $n^3$ as in problems \#1 and \#2. 
\item Express $1^5+2^5+3^5+{\dots}+n^5$ as a sum of multiples of binomial coefficients. 
\item In problem \#1, \#2, \#5, and \#7,  $n^2,n^3,n^4,$ and  $n^5$ were expressed as sums of binomial coefficients with
integer coefficients. Arrange these expressions in a Pascal-like triangle, concentrating on these integer coefficients.
The coefficients are called Eulerian numbers. 
\item Investigate the properties of this triangle. Include a discussion of row sums, how entries are found, and a
possible recursion. Use  $\left[\begin{matrix}n\\k\end{matrix}\right]$ for notation. 
\end{enumerate}
\section[PROJECT 6: THE LEIBNITZ HARMONIC TRIANGLE]{PROJECT 6: THE LEIBNITZ HARMONIC TRIANGLE}
\begin{equation*}
\frac 1 1
\end{equation*}
\ \ \ \ \ \ \ \  $\frac 1 2$\ \ \ \  $\frac 1 2$

\ \ \ \ \ \  $\frac 1 3$\ \ \ \  $\frac 1 6$\ \ \ \  $\frac 1 3$

\ \ \ \  $\frac 1 4$\ \ \ \  $\frac 1{12}$\ \ \ \  $\frac 1{12}$\ \ \ \  $\frac 1 4$

\ \  $\frac 1 5$\ \ \ \  $\frac 1{20}$\ \ \ \  $\frac 1{30}$\ \ \ \  $\frac 1{20}$\ \ \ \  $\frac 1 5$

\begin{enumerate}
\item State the rule used to form each entry, and produce the next two rows. 
\item What are the “initial conditions” needed to generate the entries?
\item Use  $\left[\begin{matrix}n\\k\end{matrix}\right]$ as the notation for entries and state properties analogous to
those found in the Pascal Triangle. Include closed formulas for each entry, row sums (alternating also), hockey stick
theorem, hexagon property, sum of squares of row entries, etc. Indicate how partial fractions help in verifying the
(infinite) hockey stick theorem.  
\item See if you can find any papers on this topic. Check PI MU EPSILON JOURNAL, THE MATHEMATICAL GAZETTE, MATHEMATICS
MAGAZINE, DELTA-UNDERGRADUATE MATHEMATICS JOURNAL, etc. 
\end{enumerate}
\section[PROJECT 7: THE TWELVE DAYS OF CHRISTMAS (in the Discrete Math Style)]{PROJECT 7: THE TWELVE DAYS OF CHRISTMAS
(in the Discrete Math Style)}
\section{}
According to the song, THE TWELVE DAYS OF CHRISTMAS the “true love” received the following number of (not so practical)
gifts:

\begin{flushleft}
\tablefirsthead{}
\tablehead{}
\tabletail{}
\tablelasttail{}
\begin{supertabular}{m{0.90875983in}m{1.3531599in}m{2.13866in}}
Day’s Gift &
Number received &
Total\\\hline
First &
1 X 12 &
12\\
Second &
2 X 11 &
22\\
Third &
3 X 10 &
30\\
Fourth &
4 X 9 &
36\\
Fifth &
5 X 8 &
40\\
Sixth &
6 X 7 &
42\\
Seventh &
7 X 6 &
42\\
Eighth &
8 X 5 &
40\\
Ninth &
9 X4 &
36\\
Tenth &
10 X 3 &
30\\
Eleventh &
11 X 2 &
22\\
Twelfth &
12 X 1 &
12\\
 &
 &
Grand Total = 364\\
\end{supertabular}
\end{flushleft}
\ \ \ \ \ \ \ \ \ \ \ \ 1

\ \ \ \ \ \ \ \ \ \ 2\ \ \ \ 2

\ \ \ \ \ \ \ \ 3\ \ \ \ 4\ \ \ \ 3

\ \ \ \ \ \ 4\ \ \ \ 6\ \ \ \ 6\ \ \ \ 4

\ \ \ \ 5\ \ \ \ 8\ \ \ \ 9\ \ \ \ 8\ \ \ \ 5

The first five rows of a triangle are given above. 

\begin{enumerate}
\item Make the next few rows. 
\item Where have you seen this triangle before?
\item Experiment with factoring each integer in the 6th row. Repeat with other rows. 
\item What are the entries in row 12?
\item Where does the expression  $1{\bullet}n+2\left(n-1\right)+3\left(n-2\right)+{\dots}+n{\bullet}1$ show up in the
triangle?
\item Explain why 
$1{\bullet}n+2\left(n-1\right)+3\left(n-2\right)+{\dots}+n{\bullet}1=\left(\begin{matrix}n+2\\3\end{matrix}\right)$.
\end{enumerate}
FOUR (OR EIGHT) DISTRIBUTION PROBLEMS

\ \ The formulas derived earlier for the number of onto functions f from a domain  $\{x_1,x_2,{\dots},x_m\}$ to a
codomain  $\{y_1,y_2,{\dots},y_n\}$ can be used as a model for the following distribution problem: in how many ways can
you place n different objects into k different bins, not allowing empty bins? Some texts use the word distinguishable
instead of different. We will also use the word similar to connote the same meaning as indistinguishable. 

\ \ The concept of compositions of n into k parts likewise serves as a model for the following distribution problem: in
how many ways can you place n similar objects into k different boxes either allowing empty boxes or not? As this second
class of problem is a little easier let’s analyze it first. 

RESULT: The number of ways of placing n similar objects into k different bins not allowing empty bins is 
$\left(\begin{matrix}n-1\\k-1\end{matrix}\right)$. Denote the k different bins by  $x_1,x_2,{\dots},x_k$. The number of
solutions to  $x_1+x_2+{\dots}+x_k=n$ where each  $x_i{\geq}1$ is the number of such distributions. 

RESULT: The number of ways of placing n similar objects into k different bins allowing empty bins is 
$\left(\begin{matrix}n+k-1\\k-1\end{matrix}\right)$. The number of solutions to  $x_1+x_2+{\dots}+x_k=n$ where each 
$x_i{\geq}0$ is the number of such distributions. Notice here that allowing,  $x_2$, for example, to be 0 is the same
as saying that no objects will be placed in bin \#2. 

\ \ Now let’s turn to the first problem. Using the Principle of Inclusion-Exclusion (PIE) we see that the number of
functions from \{1, 2, 3, 4, 5\} onto \{a, b, c\} is  $3^5-3{\bullet}2^5+3{\bullet}1^5$. Similarly, the number of
functions from \{1, 2, 3, 4, 5, 6\} onto \{a, b, c, d\} is

 
$4^6-4{\bullet}3^6+6{\bullet}2^6-4{\bullet}1^6=\left(\begin{matrix}4\\4\end{matrix}\right)4^6-\left(\begin{matrix}4\\3\end{matrix}\right)3^6+\left(\begin{matrix}4\\2\end{matrix}\right)2^6-\left(\begin{matrix}4\\1\end{matrix}\right)1^6.$
PIE can be used to show that, in general, the number of functions from  $\{x_1,x_2,{\dots},x_n\}$ onto 
$\{y_1,y_2,{\dots},y_k\}$ is

\begin{equation*}
B\left(n,k\right)=k^n-\left(\begin{matrix}k\\1\end{matrix}\right)\left(k-1\right)^n+\left(\begin{matrix}k\\2\end{matrix}\right)\left(k-2\right)^n-\left(\begin{matrix}k\\3\end{matrix}\right)\left(k-3\right)^n+{\dots}+\left(-1\right)^{k-1}\left(\begin{matrix}k\\k-1\end{matrix}\right)1^n
\end{equation*}
Each onto functions characterizes a distribution problem as follows: the following onto function 

\begin{flushleft}
\tablefirsthead{}
\tablehead{}
\tabletail{}
\tablelasttail{}
\begin{supertabular}{m{0.47055987in}|m{0.35875985in}m{0.35875985in}m{0.35875985in}m{0.29625985in}m{0.29625985in}}
x &
1 &
2 &
3 &
4 &
5\\\hline
f(x) &
a &
a &
b &
c &
b\\
\end{supertabular}
\end{flushleft}
can be considered as a model for distributing the two different objects 1 and 2 into a bin a, the two different objects
3 and 5 into bin b and 4 into bin c. Since each of the complete inverse images 
$f^{-1}\left(a\right),f^{-1}\left(b\right),$ and  $f^{-1}(c)$ is non-empty (the function being onto guarantees this)
this model counts distributions that do not allow empty bins. This establishes the following distribution result. 

RESULT: The number of ways of placing n different objects into k different bins with no empty bins is  $B(n,k)$.

RESULT: The number of ways of placing n different objects into k different bins in  $k^n.$ 

\ \ Without the restriction of “no empty bins” this is just a matter of saying that there are k  places for disposing of
each of the n objects. This is also the number of functions from   $\{x_1,x_2,{\dots},x_n\}$ to 
$\{y_1,y_2,{\dots},y_k\}$, ignoring the restriction that they must be onto. 

\ \ Finally, we look at what happens if the n objects are different but the k bins are now similar, still with no empty
bins allowed. Since there are k! ways to label the k bins, there are k! ways to convert similar bins into different
bins. The number of distributions of this type is therefore  $\frac 1{k!}B(n,k)$.  For convenience of notation, let’s
call this last expression:  $S\left(n,k\right).$ 

RESULT: The number of ways of placing n different objects into k similar bins with no empty bins is 
$S\left(n,k\right).$\ \ 

There is one additional distribution problem that we can analyze using the machinery developed thus far. In how many
ways can you place n different objects into k similar bins allowing empty bins. Using the sum rule we can look at the
following disjoint cases:

\section[\ \ \ \ \ \ No box is empty{}-  ]{\ \ \ \ \ \ No box is empty-  $S(n,k)$ }
\ \ \ \ \ \ Exactly one box is empty- $S(n,k-1)$

\section[Exactly two bins are empty{}-]{Exactly two bins are empty- $S(n,k-2)$}
\begin{equation*}
{\vdots}
\end{equation*}
\ \ \ \ \ \ All but one bin are empty- $S(n,1)$

RESULT: The number of ways of distributing n different objects into k or fewer bins is 
$S\left(n,1\right)+S\left(n,2\right)+{\dots}+S(n,k)$.

\ \ Finally, the theory of linear or integer partitions solves the final distribution problem: In how many ways can you
distribute n similar objects into k similar bins? Recall that a partition    $\left(a_1,a_2,{\dots},a_k\right)$ of an
integer n is an array of integers  $a_i$ such that  $n=a_1+a_2+{\dots}+a_n$ and 
$a_1{\geq}a_2{\geq}a_3{\geq}{\dots}{\geq}a_k>0$.  For example, the 5 partitions of n=4 are 4, 31, 22, 211, 1111. Here,
the distribution 31 is the same as 13 since the bins are similar. 

\section[DISTRIBUTIONS OF OBJECTS INTO BINS]{DISTRIBUTIONS OF OBJECTS INTO BINS}
\begin{flushleft}
\tablefirsthead{}
\tablehead{}
\tabletail{}
\tablelasttail{}
\begin{supertabular}{m{0.75115985in}m{1.6636599in}m{0.30805984in}m{1.0288599in}m{2.37266in}}
n objects &
 &
 &
k bins &
\\\hline
 &
 &
 &
Allow empty &
\ \  $P_k(n)$\\
S &
(Partitions of Integers) &
S &
 &
\\
 &
 &
 &
Do not &
  $P_k(n-k)$\\\hline
 &
 &
 &
Allow empty &
  $\left(\begin{matrix}n+k-1\\k-1\end{matrix}\right)$\\
S &
(Compositions) &
D &
 &
\\
 &
 &
 &
Do not &
  $\left(\begin{matrix}n-1\\k-1\end{matrix}\right)$\\\hline
 &
 &
 &
Allow empty &
  $S\left(n,1\right)+S\left(n,2\right)+{\dots}+S(n,k)$\\
D &
(Set Partitions) &
S &
 &
\\
 &
 &
 &
Do not &
  $\frac 1{k!}B(n,k)$\\\hline
 &
 &
 &
Allow empty &
  $k^n$\\
D &
(Surjections) &
D &
 &
\\
 &
 &
 &
Do not &
  $B(n,k)$\\\hline
\end{supertabular}
\end{flushleft}
CONVENIENT NOTATION

S = Similar

D = Different

\begin{equation*}
B\left(n,k\right)=k^n-\left(\begin{matrix}k\\1\end{matrix}\right)\left(k-1\right)^n+\left(\begin{matrix}k\\2\end{matrix}\right)\left(k-2\right)^n+{\dots}+\left(-1\right)^{k-1}\left(\begin{matrix}k\\k-1\end{matrix}\right)1^n
\end{equation*}
\begin{equation*}
S\left(n,k\right)=\frac 1{k!}B\left(n,k\right)
\end{equation*}
COMBINATORIAL PROOFS

\ \ A combinatorial proof is sometimes used to show that two very different looking expressions are in fact equal.  The
technique is as follows.  Refer to the two different looking expressions as the left-hand side (LHS) and the right-hand
side RHS.  Create a situation or question that is answered by the LHS; then show that the RHS also answers the
question.  The conclusion is that LHS=RHS.

\ \ In the following we present a number of theorems, statements, identities, etc, and give combinatorial proofs of
each.  For each result the reader is urged to attempt an alternate proof for comparison purposes.  Such an alternate
proof could be algebraic or geometric in nature; or one could try a collapsing sum or an induction proof.

THEOREM 1   $\left(\begin{matrix}2n\\2\end{matrix}\right)=2\left(\begin{matrix}n\\2\end{matrix}\right)+n^2$

\ \ Split the  $2n$ objects into two groups  A  and  B  as shown:

\ \ \ \ \ \ . . . . .\ \ \ \ \ \   . . . . . 

\ \ \ \ \ \    A\ \ \ \ \ \   B

First, you can choose 2 objects from a set of  $2n$ objects in $\left(\begin{matrix}2n\\2\end{matrix}\right)$ ways. 
Alternatively, you could select two from group  A  in $\left(\begin{matrix}n\\2\end{matrix}\right)$ ways or two from
group  B  in $\left(\begin{matrix}n\\2\end{matrix}\right)$ ways or take one from each in   $n{\bullet}n=n^2$ ways.  Now
add $\left(\begin{matrix}n\\2\end{matrix}\right)+\left(\begin{matrix}n\\2\end{matrix}\right)+n{\bullet}n$ and the
result follows.  The reader should attempt an algebraic proof using the factorial formula for
$\left(\begin{matrix}n\\k\end{matrix}\right)$.

THEOREM 2  
$\left(\begin{matrix}m+n\\2\end{matrix}\right)-\left(\begin{matrix}m\\2\end{matrix}\right)-\left(\begin{matrix}n\\2\end{matrix}\right)=\mathit{mn}$

\ \ Suppose you have a group of  m  men and  n women and you want to form men-women dancing pairs.  This can clearly be
done in mn ways.  Or, you could choose 2 from the total of m + n and delete the men-men pairs (there are 
$\left(\begin{matrix}m\\2\end{matrix}\right)$ of these) and delete the women-women pairs (also
$\left(\begin{matrix}n\\2\end{matrix}\right)$ of these).  The result follows.  The reader could attempt an algebraic
proof or perhaps a geometric proof making use of figures consisting of triangular numbers.  Also, as a challenge, the
reader could formulate a similar result involving  $\left(\begin{matrix}a+b+c\\3\end{matrix}\right)$ and a
corresponding proof.

THEOREM 3  
$\left(\begin{matrix}n\\0\end{matrix}\right)+\left(\begin{matrix}n\\1\end{matrix}\right)+\left(\begin{matrix}n\\2\end{matrix}\right)+{\dots}+\left(\begin{matrix}n\\n\end{matrix}\right)=2^n$

\ \ Here again we first create a question that is answered by either side of the given identity.  Question: How many
subsets does  $\{a_1,a_2,{\dots},a_n\}$ have?  An n-set has  $\left(\begin{matrix}n\\2\end{matrix}\right)$ subsets. 
The left-hand side counts these subsets by their size.  There are  $\left(\begin{matrix}n\\k\end{matrix}\right)$
subsets of size k.

\ \ In this situation the reader might not want to try this algebraically. 

THEOREM 4  
$\left(\begin{matrix}n\\1\end{matrix}\right)+2\left(\begin{matrix}n\\2\end{matrix}\right)+3\left(\begin{matrix}n\\3\end{matrix}\right)+{\dots}+n\left(\begin{matrix}n\\n\end{matrix}\right)=\mathit{n2}^{n-1}$

\ \ Given a set of  n  people we can select a committee of size  k  along with a chair from that committee in 
$k\left(\begin{matrix}n\\k\end{matrix}\right)$ ways.  We can select a committee (of size 1, or size 2, or …) and its
chair in 
$\left(\begin{matrix}n\\1\end{matrix}\right)+2\left(\begin{matrix}n\\2\end{matrix}\right)+3\left(\begin{matrix}n\\3\end{matrix}\right)+{\dots}+n\left(\begin{matrix}n\\n\end{matrix}\right)$
ways. Alternatively, we can explain the term  $n2^{n-1}$ as follows: choose one of the  n people to chair any of the 
$2^{n-1}$ subsets of the remaining  $n-1$ people. 

\ \ The reader is invited to investigate one or more of the following approaches:  $n2^{n-1}$ looks like a derivative,
so try differentiating  $\left(1+x\right)^n$; a reverse and add approach also works; or, first prove  
$k\left(\begin{matrix}n\\k\end{matrix}\right)=n\left(\begin{matrix}n-1\\k-1\end{matrix}\right)$ and then use it.

THEOREM 5  
$\left(\begin{matrix}n\\k\end{matrix}\right)=\left(\begin{matrix}n-1\\k\end{matrix}\right)+\left(\begin{matrix}n-1\\k-1\end{matrix}\right)$

\ \  $\left(\begin{matrix}n\\k\end{matrix}\right)$ is the number of subsets of  $\{a_1,a_2,a_3,{\dots},a_n\}$ of size k.
 Now a subset  A  of size  k either contains the fixed element $a_i$ or it does not.  If  A  contains  $a_i$, the
remaining  $k-1$ elements can be selected in  $\left(\begin{matrix}n-1\\k-1\end{matrix}\right)$ ways.  If, on the other
hand,  A  does not contain  $a_i$, you can choose the  k  elements from the depressed set  
$\left\{a_1,a_2,{\dots},a_{i-1,}a_{i+1,{\dots},}a_n\right\}$ in  $\left(\begin{matrix}n-1\\k\end{matrix}\right)$ ways. 
Since these two cases are mutually exclusive the theorem follows.  

Once again the reader is invited to attempt an algebraic proof.

THEOREM 6  Let  $d_n$ denote the number of derangements of 1, 2, 3, …, n with

 $d_0=1,d_1=0.$  Then  $d_n=\left(n-1\right)(d_{n-1}+d_{n-2})$ for  $n{\geq}2$.

In forming a derangement of 1, 2, 3, …, n  the integer  n  can be placed in any of the  $n-1$ spots 1, 2, 3, …,  $n-1$,
say spot  i.  If  i  goes into spot  n there are  $d_{n-2}$ ways to finish it.  If  i does not go into spot  n there
are  $d_{n-1}$ ways to complete the derangement.

The reader can use the principle of inclusion-exclusion to derive a formula for  $d_n$

 from which the new  recursion  $d_n=nd_{n-1}+\left(-1\right)^n$, and the above recursion, can be derived.

THEOREM 7  
$\left(\begin{matrix}n\\0\end{matrix}\right)^2+\left(\begin{matrix}n\\1\end{matrix}\right)^2+\left(\begin{matrix}n\\2\end{matrix}\right)^2+{\dots}+\left(\begin{matrix}n\\n\end{matrix}\right)^2=\left(\begin{matrix}2n\\n\end{matrix}\right)$

\ \ Given a group of  $2n$ people consisting of  n men and  n women, in how many ways can one choose a group of  n 
people?  The answer to that question is just $\left(\begin{matrix}2n\\n\end{matrix}\right)$, the right side of the
identity in question.  One could also form the group of n  people in the following way: choose  0 men and  n women in 


$\left(\begin{matrix}n\\0\end{matrix}\right)\left(\begin{matrix}n\\n\end{matrix}\right)=\left(\begin{matrix}n\\0\end{matrix}\right)^2$
ways, or, choose 1 man and  $n-1$ women in


$\left(\begin{matrix}n\\1\end{matrix}\right)\left(\begin{matrix}n\\n-1\end{matrix}\right)=\left(\begin{matrix}n\\1\end{matrix}\right)^2$
ways, or choose 2 men and  $n-2$ women in


$\left(\begin{matrix}n\\2\end{matrix}\right)\left(\begin{matrix}n\\n-2\end{matrix}\right)=\left(\begin{matrix}n\\2\end{matrix}\right)^2$
ways and so on.  Now add these disjoint cases.

\ \ An alternate algebraic proof is less interesting: Extract the coefficient of  $x^n$ from both sides of.
$\left[\left(x+1\right)^n\right]^2=\left(x+1\right)^{2n}$

THEOREM 8  
$\left(\begin{matrix}2\\2\end{matrix}\right)+\left(\begin{matrix}3\\2\end{matrix}\right)+\left(\begin{matrix}4\\2\end{matrix}\right)+{\dots}+\left(\begin{matrix}n\\2\end{matrix}\right)=\left(\begin{matrix}n+1\\3\end{matrix}\right)$

\ \ The term  $\left(\begin{matrix}n+1\\3\end{matrix}\right)$ is the number of binary strings of length  $n+1$
consisting of three 1’s (and the rest 0’s).  The left hand side counts these by where in the string the left- most 1
appears.  Let  $a_1a_2a_3{\dots}a_{n+1}$ be a string of length  $n+1$.  There are 
$\left(\begin{matrix}n\\2\end{matrix}\right)$ strings when  $a_1=1,\left(\begin{matrix}n-1\\2\end{matrix}\right)$
strings when  $a_2=1$ is the leftmost 1, …, and  $\left(\begin{matrix}2\\2\end{matrix}\right)$ strings when 
$a_{n-1}=1$ is the leftmost 1.  In this last case the string looks like 000 … 0111.

\ \ Attempting a proof by mathematical induction is an easy option.  An algebraic approach is not!

THEOREM 9  The number of positive integers that have their digits in strictly increasing order is  $2^9-1$.  Include
single digit numbers.

There are  $\left(\begin{matrix}9\\1\end{matrix}\right)$ single digit type, 
$\left(\begin{matrix}9\\2\end{matrix}\right)$ double digit type (just select 2 of the 9 digits 1, 2, 3, …, 9 and
arrange in order), …, and so on to see that there are  $\left(\begin{matrix}9\\9\end{matrix}\right)$ nine digit type.
The total is 
$\left(\begin{matrix}9\\1\end{matrix}\right)+\left(\begin{matrix}9\\2\end{matrix}\right)+\left(\begin{matrix}9\\3\end{matrix}\right)+{\dots}+\left(\begin{matrix}9\\9\end{matrix}\right)=2^9-1.$

Here is an alternative, more clever, proof.  Look at 123456789.  Any increasing number can be made by deleting any
subset of digits, except all of them.  There are  $2^9-1$ such subsets.  For example, delete the subset \{2, 4, 7, 9\}
and you get 13568.  Combining these two approaches actually gives you a nice proof that 
$\left(\begin{matrix}9\\0\end{matrix}\right)+\left(\begin{matrix}9\\1\end{matrix}\right)+\left(\begin{matrix}9\\2\end{matrix}\right)+{\dots}+\left(\begin{matrix}9\\9\end{matrix}\right)=2^9$.

THEOREM 10  
$\left(\begin{matrix}3n\\3\end{matrix}\right)=3\left(\begin{matrix}n\\3\end{matrix}\right)+6n\left(\begin{matrix}n\\2\end{matrix}\right)+n^3$

This one is a little tougher.  First rewrite as 
$n^3=\left(\begin{matrix}3n\\3\end{matrix}\right)-3\left(\begin{matrix}n\\3\end{matrix}\right)-6n\left(\begin{matrix}n\\2\end{matrix}\right)$.
 Suppose you have  n  men,  n  women and  n  children and you want to select triples consisting of one man, one woman
and one child.  There are  $n^3$ ways to do this, just pick one from each group.  Alternatively, select 3 of the 3n
people in   $\left(\begin{matrix}3n\\3\end{matrix}\right)$ ways and delete the “bad” ones.  Delete the ones where you
selected all three from one group – there are  $3\left(\begin{matrix}n\\3\end{matrix}\right)$ of these.  Now delete
those where you had two from one group and one from another – there are 
$2n\left(\begin{matrix}n\\2\end{matrix}\right)+2n\left(\begin{matrix}n\\2\end{matrix}\right)+2n\left(\begin{matrix}n\\2\end{matrix}\right)$
of these.

THEOREM 11   $1{\bullet}1!+2{\bullet}2!+3{\bullet}3!+{\dots}+n{\bullet}n!=\left(n+1\right)!-1$

In how many ways can you arrange the n+1 numbers 0, 1, 2, …, n so that they are not in ascending order?  The answer is 
$\left(n+1\right)!-1$ since 0, 1, 2, …, n is the only arrangement in ascending order.  Now lets separate into cases. 
Let  $a_0,a_1,a_2,{\dots},a_n$ represent an arrangement of these n+1 numbers.  If  $a_0{\neq}0$, there are  n  choices
left for  $a_0$, and then n! ways to fill out  $a_1,a_2,{\dots},a_n$ for a total of  $n{\bullet}n!$.  Now let  $a_0=0$
but  $a_1{\neq}1$.  There are  $n-1$ choices for  $a_1$ and  $\left(n-1\right)!$ ways to complete for a total of 
$\left(n-1\right)\left(n-1\right)!$.  Now continue with  $a_0=1,a_1=1$  but  $a_2{\neq}2$.  There are 
$\left(n-2\right)\left(n-2\right)!$ ways, and so on.

\ \ The reader should attempt a collapsing sum or induction proof. 

THEOREM 12  
$1{\bullet}n+2\left(n-1\right)+3\left(n-2\right)+{\dots}+n{\bullet}1=\left(\begin{matrix}n+2\\3\end{matrix}\right)$

\ \ Let  $S=\{0,1,2,{\dots},n,n+1\}$.  The number of subsets of S of size 3 is 
$\left(\begin{matrix}n+2\\3\end{matrix}\right)$.  Each one looks like \{a, b, c\} with a {\textless} b {\textless} c,
Let’s count these by looking at the size of the middle element  b.  If b=1 , there is one choice for a, namely a=0 and 
n  choices for  c  for a total of  $1{\bullet}n$.  If b=2 there are 2 choices for a, and  $n-1$ choices for c for a
total of  $2(n-1)$.  If b=3 the total is

  $3(n-2)$, and so on. The total derived by looking at cases is 

 $1{\bullet}n+2\left(n-1\right)+3\left(n-2\right)+{\dots}+n{\bullet}1$ and this must equal 
$\left(\begin{matrix}n+2\\3\end{matrix}\right)$ since the cases are disjoint.

THEOREM 13   $k\left(\begin{matrix}n\\k\end{matrix}\right)=n\left(\begin{matrix}n-1\\k-1\end{matrix}\right)$

\ \ Suppose you have a group of  n  people and you wish to form a subcommittee of  k  people with one of those k  people
to serve as chair. Choose the subcommittee in  $\left(\begin{matrix}n\\k\end{matrix}\right)$ ways and the chair in  k 
ways. The product rule gives $k\left(\begin{matrix}n\\k\end{matrix}\right)$ as the number of ways of selecting such a
chaired subcommittee.

\ \ Alternatively, you could first choose any one of the  n  people to serve as chair and then fill out the committee in
 $\left(\begin{matrix}n-1\\k-1\end{matrix}\right)$ ways.  There are  $n\left(\begin{matrix}n-1\\k-1\end{matrix}\right)$
ways to select a chaired subcommittee.  Hence 
$k\left(\begin{matrix}n\\k\end{matrix}\right)=n\left(\begin{matrix}n-1\\k-1\end{matrix}\right)$

\ \ The reader should attempt the easier algebraic technique by converting 
$\left(\begin{matrix}n\\k\end{matrix}\right)$ to factorial form.

THEOREM 14   $P\left(n,k\right)=k!\left(\begin{matrix}n\\k\end{matrix}\right)$

\ \ Questions – How many permutations are there of k objects chosen from a collection of  n objects?  The LHS answers
the question.  There are  $P\left(n,k\right)=n\left(n-1\right)\left(n-2\right)...(n-k+1)$ ways.  Alternatively, one
could first choose the k objects from the n objects in  $\left(\begin{matrix}n\\k\end{matrix}\right)$ ways and then
permute these in k! ways.

THEOREM 15  
$n2^{n-1}=1\left(\begin{matrix}n\\1\end{matrix}\right)+2\left(\begin{matrix}n\\2\end{matrix}\right)+3\left(\begin{matrix}n\\3\end{matrix}\right)+{\dots}+n\left(\begin{matrix}n\\n\end{matrix}\right)$

\ \ Contrast this discussion with that presented in THEOREM 4.  Look at the set of the first  $2^n$ nonnegative integers
$0,1,2,{\dots},2^n-1$. When you convert each to binary form what is the total number of 1s written?  This binary list
will look like the standard listing in  $B^n$ the set of all binary strings of length  n.  For n=3 
$B^3=\{000,001,010,011,100,101,110,111\}$. In  $B^n$ each string has length  n  and there are  $2^n$ of them.  But half
of the  $n{\bullet}2^n$ symbols are 1’s.  Then the total number of 1’s is  $n2^{n-1}$.  Alternatively, we could
consider each string and count those with one 1, then those with two 1’s, etc.  There are 
$1{\bullet}\left(\begin{matrix}n\\1\end{matrix}\right)$ with one 1,  $2\left(\begin{matrix}n\\2\end{matrix}\right)$
total 1’s in those binary numbers with exactly two 1’s,  $3\left(\begin{matrix}n\\3\end{matrix}\right)$ in those with
exactly three 1’s , and so on.  The total is 
$1\left(\begin{matrix}n\\1\end{matrix}\right)+2\left(\begin{matrix}n\\2\end{matrix}\right)+3\left(\begin{matrix}n\\3\end{matrix}\right)+{\dots}+n\left(\begin{matrix}n\\n\end{matrix}\right).$
The result now follows by equating  $n2^{n-1}$ to this sum.

THEOREM 16  
$\left(\begin{matrix}n\\0\end{matrix}\right)^2+\left(\begin{matrix}n\\1\end{matrix}\right)^2+\left(\begin{matrix}n\\2\end{matrix}\right)^2+{\dots}+\left(\begin{matrix}n\\n\end{matrix}\right)^2=\left(\begin{matrix}2n\\n\end{matrix}\right)$

\ \ Let’s revisit this identity using equivalence relations.  A binary relation R on the set of all binary strings of
length n is defined by specifying that  $\left(\alpha ,\beta \right){\in}R$ whenever weight  $\alpha =?$ weight  $\beta
$.  This R is an equivalence relation.  For n=3 there are four different equivalence classes, each containing strings
of weight 0, 1, 2, or 3.  The relation R contains  $1^2+3^2+3^2+1^2$ ordered pairs; for example, with weight 1 each of
the three strings 001, 010, 100 can be paired with any one of those same strings for a total of  $3^2=9$. These ordered
pairs can be counted in another way.  Each ordered pair looks like  $.$  Place 1’s in any three positions, and 0’s in
the others.  If you take the complement of the entries in the second coordinate an element of  R  is produced.  Here is
what one sequence of this process looks like: 

 $.$.  The reader can check that this process always produces an element of  R  and that the case of n=3 extends easily
to general n. Conclusion: choose the  n  positions for 1’s  in  $\left(\begin{matrix}2n\\n\end{matrix}\right)$ ways. 
The result follows.

THEOREM 17  
$\left(\begin{matrix}n\\0\end{matrix}\right)d_0+\left(\begin{matrix}n\\1\end{matrix}\right)d_1+\left(\begin{matrix}n\\2\end{matrix}\right)d_2+{\dots}+\left(\begin{matrix}n\\n\end{matrix}\right)d_n=n!$
where  $d_n$ denotes the  $n^{\mathit{th}}$ derangement number,  $d_0=1,d_1=0.$

\ \ The right-hand side, n!, gives the number of permutations of  n objects.  So the left-hand side must provide the
same enumeration.  The left side partitions the permutations according to how many elements are deranged (and the rest
fixed).  The term  $\left(\begin{matrix}n\\i\end{matrix}\right)d_i=\left(\begin{matrix}n\\n-i\end{matrix}\right)d_i$
gives the number of permutations of n where  $n-i$ elements are fixed and the remaining i elements are deranged. 
Summing over all i yields

\begin{equation*}
\left(\begin{matrix}n\\n\end{matrix}\right)d_0+\left(\begin{matrix}n\\n-1\end{matrix}\right)d_1+{\dots}+\left(\begin{matrix}n\\0\end{matrix}\right)d_n=\left(\begin{matrix}n\\0\end{matrix}\right)d_0+\left(\begin{matrix}n\\1\end{matrix}\right)d_1+{\dots}+\left(\begin{matrix}n\\n\end{matrix}\right)d_n=n!
\end{equation*}
THEOREM 18  
$F_{n+1}=\left(\begin{matrix}n\\0\end{matrix}\right)+\left(\begin{matrix}n-1\\1\end{matrix}\right)+\left(\begin{matrix}n-2\\2\end{matrix}\right)+{\dots}$
 where  $F_n$ denotes the  $n^{\mathit{th}}$ Fibonacci number.  

\ \ Here is a question that might resolve the issue.  How many different brick paths of length  n (and width 1) can you
make using 1 x 1 bricks and 1 x 2 bricks?  Let a(n) denote the number of such paths of length  n.  A few drawings will
show that a(1)=1, a(2)=2, a(3)=3, a(4)=5.  Since you can place a 1 x 1 brick in front of all paths of length  $n-1$ or
a 1 x 2 brick in front of all paths of length  $n-2$ we have that 
$a\left(n\right)=a\left(n-1\right)+a\left(n-2\right)$.  This recursion, along with the initial conditions, show that 
$a\left(n\right)=F_{n+1}$, the left-hand side of the identity.

\ \ Now lets look at all paths of length n and count them by the number of 1 x 2 bricks.  If there are i 1 x 2 bricks
there are  $n-i$ total bricks making up the path of length n.  Choose the positions of the  i 1 x 2 bricks in 
$\left(\begin{matrix}n-i\\i\end{matrix}\right)$ ways.  Now sum as i ranges through the values

 0, 1, 2, …, and obtain 
$\left(\begin{matrix}n\\0\end{matrix}\right)+\left(\begin{matrix}n-1\\1\end{matrix}\right)+\left(\begin{matrix}n-2\\2\end{matrix}\right)+{\dots}=F_{n+1}$.

\ \ The reader should draw all paths of length  $n=5$, for example, and examine the cases with i=0, 1, 2 . The reader
could also explore other proofs.

THEOREM 19  
$\left(\begin{matrix}m+n\\2\end{matrix}\right)-\left(\begin{matrix}m\\2\end{matrix}\right)-\left(\begin{matrix}n\\2\end{matrix}\right)=\mathit{mn}$
(Revisited)

1.\ \ \ \ \ \ \ \ \ \ \ \ \ \ \ \ \ \ \ \ \ \ 

\begin{figure}
\centering
\begin{minipage}{4.3752in}
How many lines that are not vertical or horizontal can you form by connecting the points that are on the positive  x 
and  y axes?  Since we really want only lines that are formed by connecting the  m  points with the  n  points, we
could say we have  mn lines.  Or we could consider all m +  n  points and pick 2 in 
$\left(\begin{matrix}m+n\\2\end{matrix}\right)$ ways and then delete those choices where you took  2  from  m  or  2
from n, since these formed vertical and horizontal lines, resp.; then 
$\left(\begin{matrix}m+n\\2\end{matrix}\right)-\left(\begin{matrix}m\\2\end{matrix}\right)-\left(\begin{matrix}n\\2\end{matrix}\right)=\mathit{mn}$.
\end{minipage}
\end{figure}
[Warning: Draw object ignored]  m

\begin{figure}
\centering
\begin{minipage}{0.25in}
••••
\end{minipage}
\end{figure}
[Warning: Draw object ignored]

\begin{figure}
\centering
\begin{minipage}{1in}
•  •  •  •  •
\end{minipage}
\end{figure}
 n

[Warning: Draw object ignored]

2.  There are mn one by one squares in the subdivided  

\begin{figure}
\centering
\begin{minipage}{0.3752in}
m
\end{minipage}
\end{figure}
\begin{figure}
\centering
\begin{minipage}{0.3752in}
\end{minipage}
\end{figure}
\begin{figure}
\centering
\begin{minipage}{2.1252in}
\end{minipage}
\end{figure}
[Warning: Draw object ignored]m  by  n rectangle.  Each choice of arrows 

\begin{figure}
\centering
\begin{minipage}{2.5in}
\begin{flushleft}
\tablefirsthead{}
\tablehead{}
\tabletail{}
\tablelasttail{}
\begin{supertabular}{|m{0.09415985in}|m{0.094859846in}|m{0.094859846in}|m{0.094859846in}|m{0.094859846in}|m{0.094859846in}|m{0.094859846in}|m{0.094859846in}|m{0.094859846in}|}
\hline
 &
 &
 &
 &
 &
 &
 &
 &
\\\hline
 &
 &
 &
 &
 &
 &
 &
 &
\\\hline
 &
 &
 &
 &
 &
 &
 &
 &
\\\hline
 &
 &
 &
 &
 &
 &
 &
 &
\\\hline
 &
 &
 &
 &
 &
 &
 &
 &
\\\hline
 &
 &
 &
 &
 &
 &
 &
 &
\\\hline
\end{supertabular}
\end{flushleft}
 n
\end{minipage}
\end{figure}
(one horizontal, one vertical) specifies one of these squares.  

Pick two arrows but don’t take two from the top or two from 

the side.  Then  
$\left(\begin{matrix}m+n\\2\end{matrix}\right)-\left(\begin{matrix}m\\2\end{matrix}\right)-\left(\begin{matrix}n\\2\end{matrix}\right)=\mathit{mn}$

3.  Take m people in one group and  n  in another. How many handshakes can be accomplished?  Among the m people there
are  $\left(\begin{matrix}m\\2\end{matrix}\right)$ handshakes; among the n people there are 
$\left(\begin{matrix}n\\2\end{matrix}\right)$ handshakes and between the two groups, mn.  But, 
$\left(\begin{matrix}m+n\\2\end{matrix}\right)$ also represents the total number of handshakes among the  $m+n$ people.
 Then we get: 
$\left(\begin{matrix}m\\2\end{matrix}\right)+\left(\begin{matrix}n\\2\end{matrix}\right)+\mathit{mn}=\left(\begin{matrix}m+n\\2\end{matrix}\right)$.

  [Warning: Image ignored] % Unhandled or unsupported graphics:
%\includegraphics[width=0.0098in,height=0.0161in]{w2l-img001.emf}
 
\end{document}
